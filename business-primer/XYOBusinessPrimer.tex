% Preamble
% ---
\documentclass{article}

% Packages
% ---
\usepackage{amsmath} % Advanced math typesetting
\usepackage[utf8]{inputenc} % Unicode support (Umlauts etc.)
\usepackage{hyperref} % Add a link to your document
\usepackage{graphicx} % Add pictures to your document
\usepackage{listings} % Source code formatting and highlighting
\usepackage{framed} % Source code formatting and highlighting
\usepackage{appendix} % Source code formatting and highlighting
\usepackage{csquotes} % Pretty quotes
\usepackage[automake]{glossaries}
\usepackage[letterpaper, portrait, margin=1.5in]{geometry}

\graphicspath{ {images/} }

\makeglossary

%*******************************
%**** Begin Glossary Section *****
%*******************************

\newglossaryentry{sentinel}
{
    name={Sentinel},
    description={A Sentinel is a heuristic witnesses. It observes heuristics and vouches for the certainty and accuracy of them by producing temporal ledgers. The most important aspect of a Sentinel is that it produces ledgers that Diviners can be certain came from the same source by adding Proof of Origin to them}
}



\newglossaryentry{bridge}
{
    name={Bridge},
    description={A Bridge is a heuristic transcriber. It securely relays heuristic ledgers from Sentinels to Diviners. The most important aspect of a Bridge is that a Diviner can be sure that the heuristic ledgers that are received from a Bridge have not been altered in any way. The second most important aspect of a Bridge is that they add an additional Proof of Origin metadata}
}

\newglossaryentry{archivist}
{
    name={Archivist},
    description={An Archivist stores heuristics as a part of the decentralized data set with the goal of having all historical ledgers stored, but without that requirement. Even if some data is lost or becomes temporarily unavailable, the system continues to function, just with reduced accuracy. Archivists also index ledgers so that they can return a string of ledger data if needed. Archivists store raw data only and get paid solely for retrieval of the data. Storage is always free}
}

\newglossaryentry{diviner}
{
    name={Diviner},
    description={A Diviner answers a given query by analyzing historical data that has been stored by the XYO Network. Heuristics stored in the XYO Network must have a high level of Proof of Origin to determine the validity and accuracy of the heuristic. A Diviner obtains and delivers an answer by judging the witness based on its Proof of Origin. Given that the XYO Network is a trustless system, Diviners must be incentivized to provide honest analyses of heuristics. Unlike Sentinels and Bridges, Diviners use Proof of Work to add answers to the blockchain}
}

\newglossaryentry{webble}
{
    name={webble},
    description={Everything in the world is defined spatially by its \textit{X,Y,Z,T} coordinate and nothing can leave that space. Objects are thus confined to ``webbubbles'', or what are referred to as \textit{webbles}.}
}

\newglossaryentry{gas}
{
    name={gas},
    description={The cost of a transaction (i.e. query) in the form of XYO Tokens}
}

\newglossaryentry{proof-of-origin}
{
    name={Proof of Origin},
    description={Proof of Origin is the key to verifying that ledgers flowing into the XYO Network are valid. A unique ID for source of data is not practical since it can be forged. Private Key signing is not practical since most parts of the XYO Network are difficult or impossible to physically secure, thus the potential for a bad actor to steal a Private Key is too feasible. To solve this, XYO Network uses Transient Key Chaining. The benefit of this is that it is impossible to falsify the chain of origin for data. However, once the chain is broken, it is broken forever and cannot be continued, rendering it an island}
}

\newglossaryentry{proof-of-work}
{
    name={Proof of Work},
    description={Proof of Work is a piece of data that satisfies certain requirements, is difficult to produce (i.e. costly, time-consuming), but easy for others to verify. Producing a Proof of Work can be a random process with a low probability of generation so that rigorous trial and error is required on average before a valid Proof of Work is created}
}

\newglossaryentry{bound-witness}
{
    name={Bound Witness},
    description={Bound Witness is a concept achieved by the existence of a bidirectional heuristic. Given that an untrusted source of data for the use of digital contract resolution (an oracle) is not useful, there is a substantial increase in certainty of the data provided by the establishment of such a heuristic. The primary bidirectional heuristic is proximity since both parties can validate the occurrence and range of an interaction by cosigning the interaction. This allows for a zero-knowledge proof that the two nodes were in proximity of each other.}
}

\newglossaryentry{smart-contract}
{
    name={smart contract},
    description={A protocol coined by Nick Szabo before Bitcoin, purportedly in 1994 (which is why some believe him to be Satoshi Nakamoto, the mystical and unknown inventor of Bitcoin). The idea behind smart contracts is to codify a legal agreement in a program and to have decentralized computers execute its terms, instead of humans having to interpret and act on contracts. Smart contracts collapse money (e.g. Ether) and contracts into the same concept. Being that smart contracts are deterministic (like computer programs) and fully transparent and readable, they serve as a powerful way to replace middle-men and brokers}
}

\newglossaryentry{cryptoeconomics}
{
    name={cryptoeconomics},
    description={A formal discipline that studies protocols that govern the production, distribution, and consumption of goods and services in a decentralized digital economy. Cryptoeconomics is a practical science that focuses on the design and characterization of these protocols}
}

\newglossaryentry{xyo-network}
{
    name={XYO Network},
    description={XYO Network stands for "XY Oracle Network." It is comprised of the entire system of XYO enabled components/nodes that include Sentinels, Bridges, Archivists, and Diviners. The primary function of the XYO Network is to act as a portal by which digital smart contracts can be executed through real world geo-location confirmations}
}

\newglossaryentry{certainty}
{
    name={certainty},
    description={A measure of the likelihood that a data point or heuristic is free from corruption or tampering}
}

\newglossaryentry{accuracy}
{
    name={accuracy},
    description={A measure of confidence that a data point or heuristic is within a specific margin of error}
}

\newglossaryentry{oracle}
{
    name={oracle},
    description={A part of a DApp (decentralized application) system that is responsible for resolving a digital contract by providing an answer with accuracy and certainty. The term ``oracle'' originates from cryptography where it signifies a truly random source (e.g. of a random number). This provides the necessary gate from a crypto equation to the world beyond. Oracles feed smart contracts information from beyond the chain (the real world, or off-chain). Oracles are interfaces from the digital world to the real world. As a morbid example, consider a contract for a Last Will \& Testament. A Will's terms are executed upon confirmation that the testator is deceased. An oracle service could be built to trigger a Will by compiling and aggregating relevant data from official sources. The oracle could then be used as a feed or end-point for a smart contract to call out to in order to check whether or not the person is deceased}
}

\newglossaryentry{heuristic}
{
    name={heuristic},
    description={A data point about the real world relative to the position of a Sentinel (proximity, temperature, light, motion, etc...)}
}

\newglossaryentry{transient-key-chain}
{
    name={Transient Key Chain},
    description={A Transient Key Chain links a series of data packets using Transient Key Cryptography}
}

\newglossaryentry{best-answer-score}
{
    name={Best Answer Score},
    description={A score generated by a Best Answer Algorithm that ranks the quality of the score.  The higher the score, the better it is, per the algorithm.  This score is used to determine which answer is better given two analyzed answers}
}

\newglossaryentry{best-answer-algorithm}
{
    name={Best Answer Algorithm},
    description={An algorithm used to generate Best Answer Scores when a Diviner chooses an answer.  The XYO Network permits the addition of specialized algorithms and allows the customer to specify which algorithm to use.  It is required that this algorithm will result in the same score when run on any Diviner given the same data set}
}

\newglossaryentry{origin-chain-score}
{
    name={Origin Chain Score},
    description={The score assigned to an Origin Chain to determine its credibility. This assessment takes length, tangle, overlap, and redundancy into consideration}
}

\newglossaryentry{origin-chain}
{
    name={Origin Chain},
    description={A Transient Key Chain that links together a series of Bound Witness heuristic ledger entries}
}

\newglossaryentry{origin-tree}
{
    name={Origin Tree},
    description={A data set of ledger entries taken from various Origin Chains to establish the origin of a heuristic ledger entry with a specified level of certainty}
}

\title {The XY Oracle Network Business Primer}

\author{
	Arie Trouw
		\thanks{XYO Network, \texttt{arie.trouw@xyo.network}}
	, Markus Levin
		\thanks{XYO Network, \texttt{markus.levin@xyo.network}}
	, Scott Scheper
		\thanks{XYO Network, \texttt{scott.scheper@xyo.network}}
}

\date{January 2018}

\begin{document}
\maketitle

\section{Introduction}
In 2013, a new cryptographic technology was introduced to the world: a platform called \textit{Ethereum}. A core component of Ethereum is a concept called a \gls{smart-contract}, which essentially reduces a payment and an agreement into lines of code. Imagine if a contract wasn't written on a piece of paper and signed by hand, but instead was written in computer code and executed when certain conditions were met. Smart contracts enable the world to make a leap to code that is deterministically executed by decentralized nodes running around the world.

Let us turn to the world of sports wagering. Take for instance, the following wager between two agents: \textit{Agent A} wishes to wager with \textit{Agent B} that \textit{Team A} will beat \textit{Team B} in a game. Currently, one is given no other choice but to employ a trusted disinterested third party to intermediate the transaction (in exchange for a fee). This is precisely how the eCommerce world worked before the introduction of Bitcoin. With the introduction of Ethereum, one can now program a smart contract wherein funds from the agent who bet on the losing team, are deposited automatically to the agent who bet on the winning team. One can do this by developing a smart contract to deterministically execute at a timestamp in the future (\texttt{block.timestamp}) after the game concludes. In order to determine whether \textit{Team A} or \textit{Team B} won, the contract must call out to a data source (such as a \textit{website} that lists the outcome of the game). In the world of smart contracts, this external data source is known as an \gls{oracle}. The oracle sits as the weak point in the system, as external data sources can be hacked (for instance if \textit{Agent A} secretly works for the single data source that the smart contract relied upon, thus having access to tamper with the results, he or she could manipulate the data source in order to win the wager, even if the results were different in reality. Tampering makes sense when a party is financially incentivized to do so, which is why cryptoeconomics are typically utilized to make such actions economically unviable. This example does not rely on cryptoeconomics for certainty; rather, to protect against this weak point, a concept called \textit{consensus} is deployed for oracles, wherein the smart contract does not rely on one data source, but multiple sources of data, all of which must agree and achieve consensus on the winner in order for the contract to execute. Creating such a contract enables two parties to transact in a peer-to-peer manner with their agreement and without having to use a trusted third party. It is astonishingly simple, yet up until this point in history, this was not possible. Indeed, the implications of this are profound, and though not quite noticeable today.


Since the advent of Ethereum, the cryptoassets community has experienced rapid-growth in the form of new DApps and protocol improvements. However up until this point, every platform, including Bitcoin and Ethereum, have focused almost entirely on digital channels (the online world), and not on real world channels (the offline world).

Progress has begun in the physical realm with the introduction of offline-focused cryptographic platforms that concentrate on specific use cases, such as the intersection of blockchain and the Internet of Things (\textit{IoT})\footnote{Including \textit{IOTA} (www.iota.org) and \textit{Hdac} (www.hdac.io)}. In addition, there are efforts being made to develop protocols that concentrate on the intersection of location and blockchain, which are being labeled \textit{Proof of Location}. These platforms and protocols are interesting and worth supporting; furthermore, they are useful components that serve as a spoke in the wheel of the XYO Network).

However, we still find the majority of blockchain technologies confined primarily to the narrow scope of the Internet. Since its founding in 2012, \textit{XY Findables}, the company behind the XYO Network, has been building a location network in order to make the physical world programmable and accessible to developers. In brief, XY has been working towards the concept of enabling developers (such as those writing Ethereum smart contracts) to interact with the real world as if it were an API. This undertaking is a multi-year project that requires one to separate different components into tranches.

The importance that crypto-location technologies make their way to multiple platforms should be highlighted before moving on. Thus far, all crypto-location protocols have focused on the Ethereum platform. Yet there are other compelling blockchain platforms that make strong arguments for their use, especially in specific applications. For this reason, we have built the XYO Network to be platform-agnostic from the very beginning. Our open-ended architecture ensures that the XYO Network of today will support the blockchain platforms of tomorrow. The XYO Network supports all blockchain platforms that possess smart contract execution\footnote{This includes Ethereum, Bitcoin + RSK, EOS, IOTA, NEO, Stellar, Counterparty, Monax, Dragonchain, Cardano, RChain, Lisk and others}.

Additionally, the current limitation with Proof of Location protocols (and many other blockchain DApps) centers around their complete and total reliance on Ethereum. While we believe that Ethereum will remain a critical platform in the future of blockchain technology\footnote{The XYO Network is a supporter of Vlad Zamfir's, correct-by-construction Proof of Stake consensus protocol, as well as Ethereum's \textit{sharding} clients.}, it stands imperative that end users be given the choice of what blockchain platform they wish to integrate crypto-location technologies on. Indeed, for some use cases (such as micro-transactions leveraged by IoT devices), end users may wish to use a platform that \textit{does not} charge fees for each transaction. If one is forced to use Proof of Location systems exclusively on the Ethereum platform, they must face the additional overhead of not just paying fees to use the crypto-location network, but also fees to execute the smart contract on the underlying platform.

\section{Background \& Previous Attempts}

\subsection{Proof of Location}

The concept of provable location has been around since the 1960's, and can even be dated back to the 1940's with ground-based radio-navigation systems, such as LORAN. Today there are location services that stack multiple mediums of verification on top of one another to create a Proof of Location through triangularization and GPS services. However, these approaches have yet to address the most critical component we face in location technologies today: designing a system that detects fraudulent signals and disincentivizes the faking of location data. For this reason, we propose that the most important crypto-location platform today will be the one that focuses the most on proving the origin of physical location signals.

Surprisingly, the concept of applying location verification to blockchain technologies first appeared in September 2016 at Ethereum's DevCon 2. It was introduced by Lefteris Karapetsas, an Ethereum developer from Berlin. Karapetsas' project, \textit{Sikorka}, enabled smart contracts to be deployed on the spot in the real world, using what he termed, \textit{Proof of Presence}. His application of bridging the world of blockchain and location focused primarily on augmented reality use cases; and he introduced novel concepts such as challenge questions in proving one's location.

%TODO(Christine - DONE) %https://www.reddit.com/r/ethereum/comments/539o9c/proof_of_location/
The term, Proof of Location, then formally surfaced in Ethereum's community on September 17, 2016 \cite{diferrante-proofoflocation}. It was then further expounded on by Matt Di Ferrante, Developer at the Ethereum Foundation, who stated:

\begin{displayquote}\textit{``Proof of Location you can trust is honestly one of the most difficult things to implement. Even if you have many participants that can attest each other's location, there's no guarantee that they wouldn't just go sybil at any point in the future, and since you're always only relying on majority reporting it's a huge weakness.
If you could require some type of specialized hardware device that has anti-tamper tech such that the private key is destroyed when one attempts to open it or change the firmware on it then you could possibly have greater security, but at the same time, it's not like it's impossible to spoof GPS signals either.
A proper implementation of this requires so much fallback and so many different data sources to have any assurance of accuracy, it would have to be very well funded project.''}
\vspace{2mm}
---Matt Di Ferrante
\end{displayquote}


\subsection {Proof of Location: Shortcomings}
% The Importance of Location Being Platform-agnostic:

In summary, Proof of Location can be understood as leveraging the blockchain's powerful properties, such as time-stamping and decentralization, and combining them with devices that are hard to trick. Similar to how the weakness of smart contracts centers around oracles that use a single source of truth (and thus have a single source of failure), crypto-location systems face the same problem. The weak point in crypto-location technologies centers on the devices that report back an object's location. In smart contracts, this data source is called an oracle. The core innovation surrounding the XYO Network centers around an location-based proof underlying the components of our system to create a secure crypto-location protocol.

\begin{center}
\line(1,0){50}
\end{center}

\section{The XY Oracle Network}

Location data sits quietly at the cornerstone of every part of our daily lives. Its use has increased dramatically over the last decade and now it is so ubiquitously depended on that its disappearance would be catastrophic. The reliance on location data will increase even more in the future. Tomorrow's reliance on location data will, without question, eclipse our current levels of reliance with the emergence of self-driving vehicles, package-delivering drones and smart cities that develop themselves. With the emergence of these location-reliant technologies, our lives will be in the hands of machines, and our safety will sit in direct proportion to the accuracy and validity the location the systems contain. Securing and creating a trustless source of location information will be crucial to transitioning to the world of tomorrow.

Location data has, for the most part, been provided by centralized sources of truth. History shows that such sources are susceptible to interference, vulnerable to attack and, with human-carrying technologies, can soon be fatal. Blockchain technologies, with their decentralized foundations, play a critical role in creating location-secure systems necessary for these technologies. Decentralizing location determination using a network of interconnected devices allows for a significant paradigm change in how we can source location data. Using blockchain technology to record these location transactions makes the system secure, transparent, and reliable.

Within the world of blockchain there exists smart contracts which enables the automated execution of agreements, which removes the reliance on a trusted third party to facilitate every transaction.

The data that smart contracts rely upon (called an \textit{oracle}), must have a high degree of accuracy and be verifiable. The system that records and delivers this data must be protected from the interference, attacks and error. Most importantly, the reported signals must be locked securely and publicly in time for accountability later on, which stands as perhaps blockchain's strongest properties.

%STOPPED HERE
We propose that the need for a full featured, fully decentralized and fully secured crypto-location network will be required to move us from the world of today, to the world of tomorrow. We set about achieving this with a network of technologies called The XY Oracle Network (\textit{XYO Network}). The XYO Network contains four system components, which we are detailed in this paper: Sentinels, Bridges, Archivists and Diviners. These components serve as the underpinning of an ecosystem of connected devices that enable layered location verification, across many devices, as well as many different classes of devices: Bluetooth beacons (including XY's crypto-location enabled Bluetooth device \textit{XY4+}, GPS beacons (including XY's crypto-location enabled GPS device \textit{XYGPS}), cellular, Low-Power Wide-Area Network (including XY's crypto-location enabled LoRa device \textit{XYLoRa}), smartphones, mobile applications, QR-code reading cameras, IoT devices (including doorbells, refrigerators and smart speakers) and even Low Earth Orbit ("LEO") satellites (including XY's LEO satellite, \textit{The SatoshiXY}) and more. This network of devices make it possible to determine if an object is at a specific XY-coordinate at a given time, with the most provable, trustless certainty possible. Underlying these components sits a breakthrough in IoT device security, called Proof of Origin. Tying the XYO Network's economics together are novel cryptoeconomic incentives that ensures each participant acts in accordance with the idealized-state of the XYO Network.

\textbf{We propose that the most important advancement necessary to bridging the present with the future rests on our ability to trust machines. This trust is best achieved through innovations in blockchain technology, and must be made available through the creation of a crypto-location \gls{oracle} network that is resistant to attack and achieves unprecedented \gls{accuracy} and \gls{certainty} with the given restraints of the system.} Once a location oracle network is established, all other real-world \glspl{heuristic} can be accessed as oracle data, creating a full oracle network that provides the highest confidence and accuracy possible for the technologies of tomorrow (self-driving cars, package-carrying drones, as well as others).

\subsection {Meet The Only Cryptographic Location Protocol Built For The World of Tomorrow}
With the advent of blockchain-based, trustless \glspl{smart-contract}, the need for \gls{oracle} services that arbitrate the outcome of a contract grows proportionately. Most current implementations of smart contracts rely on a single or aggregated set of authoritative oracles to settle the outcome of the contract. In cases where both parties can agree on the authority and incorruptibility of the specified oracle, this is sufficient. \textbf{However, in many cases, either an appropriate oracle does not exist or the oracle cannot be considered authoritative because of the possibility of error or corruption.}

Location oracles fall into this category. The divination of the location of a physical world item relies on the reporting, relay, storage, and processing components of the given oracle, all of which introduce error and can be corrupted. Risks include data manipulation, data pollution, data loss, and collusion. Thus the following law exists at the intersection of blockchain and location: \textit{Both \gls{certainty} and \gls{accuracy} of the location are negatively impacted by the lack of a trustless decentralized location oracle.}

\subsection{Privacy}

Similar to Bitcoin, and most blockchain technologies, the most compelling property of blockchain is the built-in accountability that arises from the fully-public ledger. This derives from the fact that each transaction is completely open and viewable. Bitcoin can be understood as a platform that is \textit{anonymous}, but not \textit{private}. The \Gls{xyo-network} shares these traditional blockchain properties; yet, being that location data is sensitive in nature, additional thought for how privacy concerns are handled becomes a necessity. For this reason the XYO Network is built with privacy at the forefront of how its platform runs.

The \Gls{xyo-network} is voluntary. Meaning, if one wishes track an item, or deploy Sentinels, Bridges or Archivists used in verifying the location of items (in exchange for XYO Token), each participant must opt-in to the network. If one does not wish to participate or have any item's location verified, then they can elect not to partipate. It is very straightforward, yet has become a rather surprisingly unique practice.  This gives one more control over their privacy compared to the choices they seem to have today. involving user privacy than what is currently used today (location tracking that is opt-out by default). It is critical that the usage of XYO Network data be voluntary because the \Gls{xyo-network} stores all the Ledger Chains as public data in the \Glspl{archivist}. This creates the possibility of inferred data that is associated with people or things to be used nefariously.

% Zero-knowledge proof explanation
The XYO Network employs a cryptographic tool called zero-knowledge proofs, which are perhaps one of the most powerful tools cryptographers have ever devised. Zero-knowledge proofs provide a method of authentication where no private data is exchanged, which means they cannot be stolen. This is novel because it provides an extra layer of security to not just information being transmitted in real-time, but also data stored on the blockchain ledger for future use.

\begin{displayquote}\textit{``Zero-knowledge proofs may be the future of private trade''}

\vspace{2mm}
---Edward Snowden
\end{displayquote}

It's important to note that location information about you, as well as your devices, is already being compiled in a non-decentralized manner; the key difference is that the data stored is \textit{not anonymous}, but tied to your identity. The XY Oracle Network focuses on making location not just trustless and decentralized, but also on being \textit{identityless}; this is achieved by combining zero-knowledge proof with a cryptographic method we call Proof of Origin, as well as other technologies (which will be covered later on).

Yet in addition to the identityless composition of the XYO Network, there is an additional layer of privacy protection in that the XYO Network is decentralized. A decentralized network removes the profit motive which can motivate malicious actors to build user profiles without permission. Since the data is publicly accessible, there's no incentive to profiteer by accessing and selling information. This is made possible due to the identityless nature of the data comprising the XYO Network ledgers.

\section{Applications}

From straightforward to complex, the \Gls{xyo-network}'s usage  has vast applications that span a multitude of industries. Take for example an eCommerce company that could offer its premium customers payment-upon-delivery services. To be able to offer this service, the eCommerce company would leverage the XYO Network and XY Platform (which uses XYO Tokens) to write a \gls{smart-contract} (i.e. on Ethereum's platform). The XYO Network could then track the location of the package being sent to the consumer along every single step of fulfillment; from the warehouse shelf to the the shipping courier, all the way into the consumer's house and every location in between. This could enable eCommerce retailers and websites to verify, in a trustless way, that the package not only appeared on the customer's doorstep, but also safely inside their home. Once the package has arrived in the customer's home (defined and verified by a specific XY-Coordinate), the shipment is considered complete and the payment to the vendor gets released. The eCommerce integration of the XYO Oracle Network thusly enables the ability to protect the merchant from fraud and ensure consumers only pay for goods that arrive in their home.

Consider an entirely different integration of the XYO Network with a hotel review site, whose current problem is that their reviews are often not trusted. Naturally, hotel owners are incentivized to improve their reviews at any cost. What if one could say with extremely high \gls{certainty} that someone was in San Diego, flew to a hotel in Bali and stayed there for two weeks, returned to San Diego, and then wrote a review about their hotel stay in Bali? The review would have a very high reputation, especially if it was written by a serial reviewer who has written many reviews with verified location data. The solutions the XYO Network can provide are infinite and the potential is unlimited.

\subsection{E-Commerce}

According to a recent study released by Comcast, more than 30\% of Americans have had a package stolen from their porch or doorstep. As the market share of eCommerce continues to grow, this problem will only become more prevalent. Megasites like Amazon are experimenting with different solutions to offer confirmed secure delivery as a premium service to their customers.

By utilizing the \Gls{xyo-network} and XYO Tokens, companies like Amazon and UPS can offer, as a premium service, an independently confirmed ledger to track every step of a shipment's progress, starting at the fulfillment center and ending with the package's secure delivery within the customer's home. As a trustless and decentralized system, the XYO Network provides independent confirmation not only of a package's delivery, but of its entire shipping history. This also allows a retailer or eCommerce site to offer payment on delivery, utilizing an Ethereum smart contract to protect the merchant from fraud or loss.

When a customer finalizes an order, an Ethereum \gls{smart-contract} is created which will release payment to the merchant upon successful delivery of the purchased product. The shipment will include an \Gls{xyo-network} \Gls{sentinel}, a low-cost electronic device that records its interactions with other devices on the XYO Network on its blockchain ledger. Other XYO Network devices will likewise record their interactions with packages being shipped. Every one of these interactions will be independently verifiable, asserting a web of locational certainty that stretches all the way back to the shipment's point of origin. When the shipment reaches its destination, as confirmed by its interaction with XYO Network devices within the buyer's home, the smart contract will be fulfilled, and payment will be released. Should there be a dispute, the ledger will provide a history that can confirm the delivery of the shipment or show where it went off track.

The terminal point of the transaction - the point where the package is delivered and payment is released - will be determined at the time the order is placed. Amazon has experimented with multiple secure delivery systems, including lockers in public places like convenience stores and even electronic locks that give their delivery team access to customers' homes. XYO Network devices within these secure locations will confirm delivery. In an Amazon locker, the shipped package will interact not just with its locker, but with XYO Network devices in other lockers and the customers that use them. In the customer's home, the XYO Network nodes could include the customer's phone, IoT devices, and even the Amazon Echo that was used to place the order.

\subsection{Hospitals and Medical Errors}
Medical errors are the third leading cause of death in the United States, according to \href{https://www.usnews.com/news/articles/2016-05-03/medical-errors-are-third-leading-cause-of-death-in-the-us}{a study released by the Johns Hopkins School of Medicine}. Many of these preventable deaths are a result of operational or record-keeping errors, including adverse drug interactions, improper medical records, and even unnecessary surgeries. The study's author, Dr. Martin Makary, said in a letter to the Centers for Disease Control and Prevention that ``it is time for the country to invest in medical quality and patient safety proportional to the mortality burden it bears. This would [include] research in technology that reduces harmful and unwarranted variation in medical care.''

\begin{displayquote}\textit{``It is time for the country to invest in medical quality and patient safety proportional to the mortality burden it bears. This would [include] research in technology that reduces harmful and unwarranted variation in medical care.''}
\vspace{2mm}
---Dr. Martin Makary
\end{displayquote}

By tying the \Gls{xyo-network} into Hospitals' operational frameworks that are already in place, care providers can significantly reduce failures in communication and record keeping that result in patient injury and death. Utilizing the XYO Network and XYO Tokens can provide a trustless, decentralized, and independently verifiable record of all patient interactions with any staff as well as a log of relevant patient data such as the patient?s vitals, treatment details, and test results for the duration of their stay.

The \Gls{xyo-network} is a web of devices that record and archive \gls{heuristic} data using a blockchain ledger. Whenever a device on the XYO Network interacts with another XY device, it logs this interaction. By reviewing this ledger of interactions and the additional data it  provides, it is possible to verify with a high degree of \gls{certainty} that a specific interaction happened at a specific time in a specific location.

For example, imagine a patient, John Doe, who is admitted to the E.R. John is given an identifying bracelet that is also an \Gls{xyo-network} \Gls{sentinel}, which keeps a record of any XY devices John interacts with. The monitor that reads John?s vital signs is also an XY Sentinel. It logs John's vitals as heuristic data, and the communication between the two devices eliminates the potential for human error in record keeping. The monitor also serves as an XYO Network \Gls{bridge}, reporting and archiving the blockchain ledgers of any XY Sentinels it interacts with.

When John is treated by a doctor or nurse, these interactions are recorded on John's ledger, the monitor's ledger, and the ledger of an XY \Gls{sentinel} embedded in the staff member's hospital ID. The XYO Network could even keep a log of medications John receives, and because a Sentinel could be linked to the medication itself, it could provide confirmation that the correct dosage of the correct medication was administered, confirming the accuracy of John's medical record.


\section{XY Findables}

The XYO Network will be built upon an existing infrastructure of 1,000,000 devices that we have distributed throughout the world via our consumer-facing business, XY - The Findables Company. XY's Bluetooth and GPS devices allow everyday consumers to place physical tracking beacons on the things they want to keep track of (such as keys, luggage, bikes and even pets). If they misplace or lose such an item, they can see exactly where it is by viewing its location on a smartphone application. In just six years, XY has created one of the largest consumer Bluetooth and GPS networks in the world.

We are fortunate to have a consumer business that has successfully built this real-world network. Most location networks fail to reach this phase and get the critical mass necessary to building out an extensive network. However the Sentinel network we have created is only the starting point. The XYO Network is an open system that any operator of location devices can plug into the XYO Network and begin earning XYO Tokens.

To further grow its network, the \Gls{xyo-network} is engaging with businesses to expand its network of \Glspl{sentinel} beyond its own network of XY beacons. A variety of devices can act as Sentinels. Generally, the greater the Sentinel cardinality in the XYO Network, the more reliable the network.


\section{Our Team}

%TODO: Christine: Add the previous Team section here

\section{Token Economy}

The XYO Network will rely on an ERC20 token called the XYO Token used to incentivize the desired behavior of providing accurate, reliable location. XYO Tokens can be thought of as``gas'' needed to interface with the real world in order to verify the XY-coordinate of a specified object.

The process works like this: A token holder queries the XYO Network with a query, (e.g. ``Where is my eCommerce order package with XYO Address \texttt{0x123456789}''). The query gets sent into a queue, where it waits to be processed and answered. A user can set their desired confidence level and XYO gas cost at query creation. The cost of a query (in XYO tokens) is determined by the amount of data required to provide an answer to the query and will depend on market dynamics.  The more data needed, the more expensive the query and higher the XYO Gas Price. Potential queries to the XYO Network can be very large and expensive. For instance, a trucking and logistics company could query the XYO Network to ask, \textit{``What is the location of every single car in our fleet?''}

Once the XYO Token holder queries the XYO Network and pays the requested gas, Diviners taking hold of the task call out to the relevant \Gls{archivist} to retrieve the pertinent data needed to answer the query. The data returned is derived from the \Glspl{bridge}, who originally gathered the data from the \Glspl{sentinel}. Sentinels are essentially devices or signals that verify the location of objects. These include devices like Bluetooth trackers, GPS trackers, geo-location tracking built into IoT devices, satellite tracking technology, QR-code scanners, RFID scanning and many others. XY Findables has pioneered and launched its consumer business, XY Find It, which has allowed it to test and process real-world location heuristics. All efforts of XY Find It have served to help significantly in designing the XYO Network Blockchain Protocol.

\section{Token Generation Event}

As part of our launch, the XYO Network will create a Token Generation Event, when we distribute the first instances of XYO tokens that can be used to power queries in our platform.

\subsection {XYO Token Specifications}
\begin{itemize}
\item Smart contract platform: Ethereum
\item Contract Type: ERC-20
\item Token: XYO
\item Token Name: \Gls{xyo-network} Utility Token
\item Token Address: 0x55296f69f40ea6d20e478533c15a6b08b654e758
\item Total issuance: Finite and capped at the amount reached after the Token Main Sale
\item Amount issued during the main sale: Unlimited (Burned after the token sale event. No further XYO tokens will be generated after the Main Sale ends)
\end{itemize}

\section{Roadmap}

%TODO(Christine): Add the RoadMap section from here 2017 on down

\section{Cryptoeconomics}

The elephant in the room with cryptoeconomics today is that so many coins have become more useless than the assets they were seeking to displace (fiat currencies).

We hold that the value of a coin lays within direct proportion to its utility, which to some degree relies on the number of transactions it experiences. Many cryptocurrencies today focus almost exclusively on incentive-systems for rewarding miners; they do not focus on building incentives for token users. As a consequence, with enough time, this imbalance creates a less than ideal ecosystem for every participant involved (miners, token holders and tertiary entities who build upon its platform).

There is a natural ratio of liquidity in any healthy economic token system. In far too many, the pendulum rests frozen in time at one end of the spectrum. In the case of Bitcoin and even Ethereum, where a small minority of mining pools control the majority of the ecosystem, this creates a problem each system aims to solve: centralization.

\subsection{Incentivizing Token Usage}

A system in which the holder is encouraged to \textit{not} use their tokens rather than transacting with their tokens creates a long-term problem for the underlying economy.  It creates an ecosystem with very scarce stores of value and triggers a natural impulse to invent reasons for \textit{not} using the coin instead of creating utility and liquidity. Lack of token liquidity is often ignored by token holders because the artificial scarcity created by reticent token-spending creates short-term spikes, but the question is: at what cost?

\textbf{The problem of most cryptoeconomic incentives centers on their focus. The focus is placed too strongly on the token miners, and not at all on the token users. The XYO Token to take both into account by defining the ideal-state and rewarding the market participants who hold in-memory accounts of the ideal-state and act upon reaching it.}

At different points in the token's economy, the user is rewarded with different token reward mechanisms depending on the natural flow of the system as part of XYO's liquidity incentives. In a system where transactions are rampant, a user who preserves the token won't be missing out on transacting. However just as safety measures are put in place for preventing fraud amongst miner coming up with the wrong answers, by punishing them with XYO Token loss, so too will users be punished who are trying to game the system by transacting with other parties in a circular manner in hopes of gaming the liquidity incentives.

On the miner incentive side, there will be mechanisms introduced where the XYO Miner is incentivized to not just provide accurate data, but also to know when to \textit{not} provide data at all, passing off the opportunity to a competing XYO Miner (Sentinel, Archivist, etc.). 

For XYO Token holders, one will be rewarded to transact more when network liquidity is low, compared to when network liquidity is high. Similar to how credit cards reward consumers for spending money, in this model the consumer receives rewards which are given up by XYO Miners who could have computed or verified the data, but elected not to for the health of the ecosystem. Essentially, the rich machines forfeit the reward they would have received and passed it on to the end-user transacting (as well as the second-best machine that took its place for not acting), in order to create a better token system. Thus proving that it has the best intent of the token ecosystem in mind.

The Bitcoin mining market presents a situation similar to the \textit{prisoner's dilemma}. Bitcoin, as a whole, would benefit more if market participants collaborated to some degree; however, in system design, absolutes typically prevail due to simplicity. Adam Smith, calls this \textit{greatest exactness}, declaring ``accurate in the highest degree, and admit of no exceptions or modifications, but such as may be ascertained as accurately as the rules themselves, and which generally, indeed, flow from the very same principles with them.''\cite{adamsmith-sentiments} For economies that rely on human actors, subject to human nature, simplistic hard rules prevail. Smith understood that it is human nature to operate with \textit{absolute} rules rather than rules of \textit{moderation} because holding the ideal-state of the system concurrently in-memory is taxing to the brain. In other words, ``hard-and-fast rules are easier to keep than rules that are slightly relaxed. The opposite should be true.''\cite{roberts-howadamsmith} As a result, current cryptocurrency token economies are inefficient as tokens do not incentivize participants properly, partly because they are based on economic theory that pre-date blockchain technologies.

TODO: ADD THE IDEA FOR THE LOTTERY

\begin{center}
\line(1,0){50}
\end{center}

\begin{thebibliography}{9}
\bibitem{roberts-howadamsmith}
Russ Roberts.
\textit{How Adam Smith Can Change Your Life}.
Portfolio / Penguin, New York, NY, 2014.

\bibitem{numeraire}
Richard Craib, Geoffrey Bradway, Xander Dunn with Joey Krug
\textit{Numeraire: A Cryptographic Token for Coordinating Machine Intelligence and Preventing Overfitting}.
\\\texttt{https://numer.ai/whitepaper.pdf}
February 20, 2017.

\bibitem{adamsmith-sentiments}
Adam Smith
\textit{The Theory of Moral Sentiments}.
A. Millar, London, 1759.


\end{thebibliography}

%*******************************
%**** End Glossary Section *****
%*******************************

\end{document}
