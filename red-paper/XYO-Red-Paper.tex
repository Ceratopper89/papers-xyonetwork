% Preamble
% ---
\documentclass{article}

% Packages
% ---
\usepackage{amsmath} % Advanced math typesetting
\usepackage[utf8]{inputenc} % Unicode support (Umlauts etc.)
\usepackage{hyperref} % Add a link to your document
\usepackage{graphicx} % Add pictures to your document
\usepackage{listings} % Source code formatting and highlighting
\usepackage{framed} % Source code formatting and highlighting
\usepackage{appendix} % Source code formatting and highlighting
\usepackage{csquotes} % Pretty quotes
\usepackage[automake]{glossaries}
\usepackage{xcolor}
\usepackage{pagecolor}
\usepackage[letterpaper, portrait, margin=1.5in]{geometry}

\graphicspath{ {images/} }

\makeglossary

%*******************************
%**** Begin Glossary Section *****
%*******************************
\newglossaryentry{sentinel}
{
    name={Sentinel},
    description={A Sentinel is a heuristic witnesses. It observes heuristics and vouches for the certainty and accuracy of them by producing temporal ledgers. The most important aspect of a Sentinel is that it produces ledgers that Diviners can be certain came from the same source by adding Proof of Origin to them}
}

\newglossaryentry{bridge}
{
    name={Bridge},
    description={A Bridge is a heuristic transcriber. It securely relays heuristic ledgers from Sentinels to Diviners. The most important aspect of a Bridge is that a Diviner can be sure that the heuristic ledgers that are received from a Bridge have not been altered in any way. The second most important aspect of a Bridge is that they add an additional Proof of Origin metadata}
}

\definecolor{lightred}{rgb}{1,0.85,0.85}

\title {XYO Network: Security Risks and Mitigants}

\author{
    Arie Trouw
        \thanks{XYO Network, \texttt{arie@xyo.network}}
}

\date{February 2018 - Early Draft}

\begin{document}

\pagecolor{lightred}

\maketitle

\begin{center}
\line(1,0){50}
\end{center}

\section{Introduction}
A primary concern for the XYO Network, like all decentralized trustless systems, is the security of the system.  Vulnerabilities include, but are not limited to, design/architecture flaws, coding errors, incorrect economic motivation, and social engineering. For the purposes of this document, we will focus on design/architecture flaws and economic motivation.

\section{Poison the Well Attacks}
\subsection{Summary}
A Poison the Well Attack is where a malfunctioning or malicious party is creating corrupt data which decreases the accuracy and/or certainty of results generated by the system.

\subsection{Motivation}
In this attack the bad actor is motivated to disrupt or \emph{poison} the data that is going to a particular \gls{sentinel} or \gls{bridge}. This could allow them to cause both short term and long term financial disruption. The XYO Network is a trustless system which gives it a low tolerance for this bad data. 

While there is no direct gain for the bad actor, often there are benefits to disrupting other people's financial and reputation standing. For instance, let's surmise that the XYO Network is being used to track parolees locations to monitor for any infractions they commit by being somewhere against their parole terms. This could be as simple as not allowing a continual DUI offender to be in a bar past a certain amount of time. An enterprising parolee could poison the well in terms of feeding the bridge for the bar bad data until it is knocked off the network. Then they could come and go from the bar as they please. Even if the data showed them locating in the bar, the bad data could limit the ability to prosecute.

\subsection{Technical Analysis}
Location data of the sentinel could be affected by a GPS Jammer or illegal radio frequency transmitters that are designed to interfere with authorized radio communications. GPS Spoofing devices \cite{gps1} can send false data to GPS receiving radios to falsify their location. 

Due to the Sentinels communicating via Bluetooth, this presents another vector for this attack. They are various documented ways to spoof a bluetooth device to send bad data \cite{bluetooth1}. While the private keys are ephemeral, it is possible that a device could listen to the sentinel communicating with the bridge and copy the data it is sending across. Theoretically it could then send bad data as that sentinel which would begin to poison the data the bridge is sending to the archiver. 

\subsection{Mitigation Strategies}
While effective, a GPS Jammer is easily recognized due to the pollution it causes to the general area. For instance, anyone will a cell phone in the area would notice a complete block out of many of the apps they use. It would be a fairly short amount of time until it was discovered that there was a jammer in the area. Given that the disruption is easily noticed and the FCC has explicit ruling that jammers are illegal \cite{fcc1}, the likelihood of this attack is low. That being said, there are sophisticated GPS anti-spoofing techniques at the hardware and software level being developed. \cite{gps1}

There are current technologies and strategies that allow us to protect against bluetooth spoofing and disruption, such as authenticated link key using secure connections. \cite{bluetooth2}

\section{Assassination Attacks}
An Assassination Attack is where a malicious actor tries to discredit (character assassination) or make non-functional (technical assassination) another node.

\section{Deception Attacks}
A Deception Attack is where a malicious actor tries to pass off incorrect yet valid data to be used in the system for personal gain.

\section{Denial of Service Attacks}
A Denial of Service Attack (DoS) is when a malicious or dysfunctional actor causes a local, regional, or system wide outage.

\section{Revision Attacks}
A Revision Attack is when a malicious or dysfunctional actor causes historically immutable data to mutate.

\begin{thebibliography}{9}

\bibitem{gps1}
Ali Jafarnia-Jahromi, Ali Broumandan, John Nielsen, and Gerard Lachapelle.
\textit{GPS Vulnerability to Spoofing Threats and a Review of Antispoofing Techniques}.
International Journal of Navigation and Observation, vol. 2012.
\textit{Hindawai.com}.
\\\texttt{https://www.hindawi.com/journals/ijno/2012/127072/cta/}

\bibitem{bluetooth1}
John Padgette John Bahr Mayank Batra Marcel Holtmann Rhonda Smithbey Lily Chen Karen Scarfone
\textit{Guide to Bluetooth Security}.
NIST 2017.
\textit{NIST Special Publication}.
\\\texttt{http://nvlpubs.nist.gov/nistpubs/SpecialPublications/NIST.SP.800-121r2.pdf}

\bibitem{bluetooth2}
JP Dunning
\textit{Breaking Bluetooth by being bored}.
DefCon 2010.
\textit{Defcon Presentation}.
\\\texttt{https://www.defcon.org/images/defcon-18/dc-18-presentations\-/Dunning/DEFCON-18-Dunning-Breaking-Bluetooth.pdf}

\bibitem{fcc1}
FCC.
\textit{FCC Enforcement Advisory}.
FCC 2014.
\textit{FCC.gov}.
\\\texttt{https://apps.fcc.gov/edocs\_public/attachmatch/DA-14-1785A1.pdf}


\end{thebibliography}
\printglossaries

%*******************************
%**** End Glossary Section *****
%*******************************

\end{document}
