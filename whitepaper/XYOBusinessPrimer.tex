% Preamble
% ---
\documentclass{article}

% Packages
% ---
\usepackage{amsmath} % Advanced math typesetting
\usepackage[utf8]{inputenc} % Unicode support (Umlauts etc.)
\usepackage{hyperref} % Add a link to your document
\usepackage{graphicx} % Add pictures to your document
\usepackage{listings} % Source code formatting and highlighting
\usepackage{framed} % Source code formatting and highlighting
\usepackage{appendix} % Source code formatting and highlighting
\usepackage{csquotes} % Pretty quotes
\usepackage[automake]{glossaries}
\usepackage[letterpaper, portrait, margin=1.5in]{geometry}

\graphicspath{ {images/} }

\makeglossary

%*******************************
%**** Begin Glossary Section *****
%*******************************

\newglossaryentry{sentinel}
{
    name={Sentinel},
    description={A Sentinel is a heuristic witnesses. It observes heuristics and vouches for the certainty and accuracy of them by producing temporal ledgers. The most important aspect of a Sentinel is that it produces ledgers that Diviners can be certain came from the same source by adding Proof of Origin to them}
}



\newglossaryentry{bridge}
{
    name={Bridge},
    description={A Bridge is a heuristic transcriber. It securely relays heuristic ledgers from Sentinels to Diviners. The most important aspect of a Bridge is that a Diviner can be sure that the heuristic ledgers that are received from a Bridge have not been altered in any way. The second most important aspect of a Bridge is that they add an additional Proof of Origin metadata}
}

\newglossaryentry{archivist}
{
    name={Archivist},
    description={An Archivist stores heuristics as a part of the decentralized data set with the goal of having all historical ledgers stored, but without that requirement. Even if some data is lost or becomes temporarily unavailable, the system continues to function, just with reduced accuracy. Archivists also index ledgers so that they can return a string of ledger data if needed. Archivists store raw data only and get paid solely for retrieval of the data. Storage is always free}
}

\newglossaryentry{diviner}
{
    name={Diviner},
    description={A Diviner answers a given query by analyzing historical data that has been stored by the XYO Network. Heuristics stored in the XYO Network must have a high level of Proof of Origin to determine the validity and accuracy of the heuristic. A Diviner obtains and delivers an answer by judging the witness based on its Proof of Origin. Given that the XYO Network is a trustless system, Diviners must be incentivized to provide honest analyses of heuristics. Unlike Sentinels and Bridges, Diviners use Proof of Work to add answers to the blockchain}
}

\newglossaryentry{webble}
{
    name={webble},
    description={Everything in the world is defined spatially by its \textit{X,Y,Z,T} coordinate and nothing can leave that space. Objects are thus confined to ``webbubbles'', or what are referred to as \textit{webbles}.}
}

\newglossaryentry{gas}
{
    name={gas},
    description={The cost of a transaction (i.e. query) in the form of XYO Tokens}
}

\newglossaryentry{proof-of-origin}
{
    name={Proof of Origin},
    description={Proof of Origin is the key to verifying that ledgers flowing into the XYO Network are valid. A unique ID for source of data is not practical since it can be forged. Private Key signing is not practical since most parts of the XYO Network are difficult or impossible to physically secure, thus the potential for a bad actor to steal a Private Key is too feasible. To solve this, XYO Network uses Transient Key Chaining. The benefit of this is that it is impossible to falsify the chain of origin for data. However, once the chain is broken, it is broken forever and cannot be continued, rendering it an island}
}

\newglossaryentry{proof-of-work}
{
    name={Proof of Work},
    description={Proof of Work is a piece of data that satisfies certain requirements, is difficult to produce (i.e. costly, time-consuming), but easy for others to verify. Producing a Proof of Work can be a random process with a low probability of generation so that rigorous trial and error is required on average before a valid Proof of Work is created}
}

\newglossaryentry{bound-witness}
{
    name={Bound Witness},
    description={Bound Witness is a concept achieved by the existence of a bidirectional heuristic. Given that an untrusted source of data for the use of digital contract resolution (an oracle) is not useful, there is a substantial increase in certainty of the data provided by the establishment of such a heuristic. The primary bidirectional heuristic is proximity since both parties can validate the occurrence and range of an interaction by cosigning the interaction. This allows for a zero-knowledge proof that the two nodes were in proximity of each other.}
}

\newglossaryentry{smart-contract}
{
    name={smart contract},
    description={A protocol coined by Nick Szabo before Bitcoin, purportedly in 1994 (which is why some believe him to be Satoshi Nakamoto, the mystical and unknown inventor of Bitcoin). The idea behind smart contracts is to codify a legal agreement in a program and to have decentralized computers execute its terms, instead of humans having to interpret and act on contracts. Smart contracts collapse money (e.g. Ether) and contracts into the same concept. Being that smart contracts are deterministic (like computer programs) and fully transparent and readable, they serve as a powerful way to replace middle-men and brokers}
}

\newglossaryentry{cryptoeconomics}
{
    name={cryptoeconomics},
    description={A formal discipline that studies protocols that govern the production, distribution, and consumption of goods and services in a decentralized digital economy. Cryptoeconomics is a practical science that focuses on the design and characterization of these protocols}
}

\newglossaryentry{xyo-network}
{
    name={XYO Network},
    description={XYO Network stands for "XY Oracle Network." It is comprised of the entire system of XYO enabled components/nodes that include Sentinels, Bridges, Archivists, and Diviners. The primary function of the XYO Network is to act as a portal by which digital smart contracts can be executed through real world geo-location confirmations}
}

\newglossaryentry{certainty}
{
    name={certainty},
    description={A measure of the likelihood that a data point or heuristic is free from corruption or tampering}
}

\newglossaryentry{accuracy}
{
    name={accuracy},
    description={A measure of confidence that a data point or heuristic is within a specific margin of error}
}

\newglossaryentry{oracle}
{
    name={oracle},
    description={A part of a DApp (decentralized application) system that is responsible for resolving a digital contract by providing an answer with accuracy and certainty. The term ``oracle'' originates from cryptography where it signifies a truly random source (e.g. of a random number). This provides the necessary gate from a crypto equation to the world beyond. Oracles feed smart contracts information from beyond the chain (the real world, or off-chain). Oracles are interfaces from the digital world to the real world. As a morbid example, consider a contract for a Last Will \& Testament. A Will's terms are executed upon confirmation that the testator is deceased. An oracle service could be built to trigger a Will by compiling and aggregating relevant data from official sources. The oracle could then be used as a feed or end-point for a smart contract to call out to in order to check whether or not the person is deceased}
}

\newglossaryentry{heuristic}
{
    name={heuristic},
    description={A data point about the real world relative to the position of a Sentinel (proximity, temperature, light, motion, etc...)}
}

\newglossaryentry{transient-key-chain}
{
    name={Transient Key Chain},
    description={A Transient Key Chain links a series of data packets using Transient Key Cryptography}
}

\newglossaryentry{best-answer-score}
{
    name={Best Answer Score},
    description={A score generated by a Best Answer Algorithm that ranks the quality of the score.  The higher the score, the better it is, per the algorithm.  This score is used to determine which answer is better given two analyzed answers}
}

\newglossaryentry{best-answer-algorithm}
{
    name={Best Answer Algorithm},
    description={An algorithm used to generate Best Answer Scores when a Diviner chooses an answer.  The XYO Network permits the addition of specialized algorithms and allows the customer to specify which algorithm to use.  It is required that this algorithm will result in the same score when run on any Diviner given the same data set}
}

\newglossaryentry{origin-chain-score}
{
    name={Origin Chain Score},
    description={The score assigned to an Origin Chain to determine its credibility. This assessment takes length, tangle, overlap, and redundancy into consideration}
}

\newglossaryentry{origin-chain}
{
    name={Origin Chain},
    description={A Transient Key Chain that links together a series of Bound Witness heuristic ledger entries}
}

\newglossaryentry{origin-tree}
{
    name={Origin Tree},
    description={A data set of ledger entries taken from various Origin Chains to establish the origin of a heuristic ledger entry with a specified level of certainty}
}

\title {The XY Oracle Network Business Primer}

\author{
	Arie Trouw
		\thanks{XYO Network, \texttt{arie.trouw@xyo.network}}
	, Markus Levin
		\thanks{XYO Network, \texttt{markus.levin@xyo.network}}
	, Raul Jordan
		\thanks{Harvard College, The Thiel Foundation and XYO Network, \texttt{rauljordan@college.harvard.edu}}
	, Scott Scheper
		\thanks{XYO Network, \texttt{scott.scheper@xyo.network}}
}

\date{January 2018}

\begin{document}
\maketitle

\begin{center}
\line(1,0){50}
\end{center}


%Introduction
\section{Introduction}
On January 3, 2009, the world changed forever. The Bitcoin network was born after its first block, the "genesis block", was mined. The brilliance of Bitcoin centered around one simple, yet profound concept: the elimination of trust. At that point, commerce on the Internet relied on financial institutions who served as trusted third parties to facilitate every single transaction. Bitcoin enabled humans who didn't already know one another to transact in peer-to-peer fashion over a digital communications channel. But the most novel component of Bitcoin centered around the way it constructed a solution to the double-spending problem. This was made possible due to leveraging \gls{cryptoeconomics} with a tool called \gls{proof-of-work} and utilizing a system of chained blocks known as blockchain. For the first time in history, humans were able to safely transact without knowing one another over a digital communications channel, and without an intermediary financial institution. Today, Bitcoin has gone from a little-known concept to one attracting mainstream popularity.

Yet, in 2013 a new cryptographic technology was introduced to the world which caused even more excitement: a platform called \textit{Ethereum}. Ethereum introduced a Turing-complete programming language into blockchain technology. It's somewhat helpful to think of Ethereum like this: Imagine Bitcoin as a payment app on your phone. Think of Ethereum as the phone's entire operating system which enables one to create apps that can do virtually anything.
A core component of Ethereum is a concept called a \gls{smart-contract}, which essentially collapses a payment and an agreement into the same thing. Imagine if a contract wasn't written on a piece of paper and signed by hand, but instead was written in computer code and executed when certain conditions were met. Smart contracts enable the world to make a leap from "wet code" on pieces of paper, to "dry code" that get deterministically executed by Ethereum computer nodes that are decentralized and are running around the world.

Let us turn to games, but in part to avoid \textit{Ludic fallacy} we will not look to games of cards or dice, but instead look to the world of sports wagering, which serves as a nice illustration of how smart contracts can be deployed. Take for instance, the following wager between two agents: \textit{Agent A} wishes to wager with \textit{Agent B} that \textit{Team A} will beat \textit{Team B} in a game. Currently, one is given no other choice but to employ a trusted disinterested third party to intermediate the transaction (in exchange for a fee). This is precisely how the eCommerce world worked before the introduction of Bitcoin. With the introduction of Ethereum, one can now program a smart contract wherein funds from the agent who bet on the losing team, are deposited automatically to the agent who bet on the winning team. One can do this by developing a smart contract to deterministically execute at a timestamp in the future (\texttt{block.timestamp}) after the game concludes. In order to determine whether \textit{Team A} or \textit{Team B} won, the contract must call out to a data source (such as a \textit{website} that lists the outcome of the game). In the world of smart contracts, this external data source is known as an \gls{oracle}. The oracle sits as the weak point in the system, as external data sources can be hacked (for instance if \textit{Agent A} secretly works for the single data source that the smart contract relied upon, thus having access to tamper with the results, he or she could manipulate the data source in order to win the wager, even if the results were different in reality. Tampering makes sense when a party is financially incentivized to do so, which is why cryptoeconomics are typically utilized to make such actions economically unviable. This example does not rely on cryptoeconomics for certainty; rather, to protect against this weak point, a concept called \textit{consensus} is deployed for oracles, wherein the smart contract does not rely on one data source, but multiple sources of data, all of which must agree and achieve consensus on the winner in order for the contract to execute. Creating such a contract enables two parties to transact in a peer-to-peer manner with their agreement and without having to use a trusted third party. It is astonishingly simple, yet up until this point in history, this was not possible. Indeed, the implications of this are profound, and though not quite noticeable today, decentralized deterministic contracts will be omnipresent in the world of tomorrow.

Now that smart contracts have been illustrated, it is important to note another component that Ethereum introduced to the world: decentralized applications, referred to as \textit{DApps}. DApps enable developers to create applications that do not rely on centralized sources. They run exactly as programmed without any possibility of downtime, censorship, fraud or third-party interference. DApps run on specialized blockchains, which are enormously powerful, globally shared infrastructures.

Since the advent of Ethereum, the cryptographic community has experienced rapid-growth in the form of new DApps and protocol improvements. However up until this point, every platform, including Bitcoin and Ethereum, have focused almost entirely on digital channels (the online world), and not on real world channels (the offline world).

Thankfully, progress has begun in the offline realm with the introduction of offline-focused cryptographic platforms that concentrate on specific use cases, such as the intersection of blockchain and the Internet of Things (\textit{IoT})\footnote{Including \textit{IOTA} (www.iota.org) and \textit{Hdac} (www.hdac.io)}. In addition, there are efforts being made to develop protocols that concentrate on the intersection of location and blockchain, which are being labeled \textit{Proof of Location}. These platforms and protocols are interesting and worth supporting; furthermore, they are useful components that serve as a spoke in the wheel of the XYO Network).

Yet we still find the majority of blockchain technologies confined primarily to the narrow scope of the Internet. Since its founding in 2012, \textit{XY}, the company behind the XYO Network, has been building a location network in order to make the physical world programmable and accessible to developers. In brief, XY has been working towards the concept of enabling developers (such as those writing Ethereum smart contracts) to interact with the real world as if it were an API. This undertaking is a multi-year project that requires one to separate different components into tranches.

The importance that crypto-location technologies make their way to multiple platforms should be highlighted before moving on. Thus far, all crypto-location protocols have focused on the Ethereum platform. Yet there are other compelling blockchain platforms that make strong arguments for their use, especially in specific applications. For this reason, we have built the XYO Network to be platform-agnostic from the very beginning. Our open-ended architecture ensures that the XYO Network of today will support the blockchain platforms of tomorrow. The XYO Network supports all blockchain platforms that possess smart contract execution\footnote{This includes Ethereum, Bitcoin + RSK, EOS, IOTA, NEO, Stellar, Counterparty, Monax, Dragonchain, Cardano, RChain, Lisk and others}.

Additionally, the current limitation with Proof of Location protocols (and many other blockchain DApps) centers around their complete and total reliance on Ethereum. While we believe that Ethereum will remain a critical platform in the future of blockchain technology\footnote{The XYO Network is a supporter of Vlad Zamfir's, correct-by-construction Proof of Stake consensus protocol, as well as Ethereum's \textit{sharding} clients.}, it stands imperative that end users be given the choice of what blockchain platform they wish to integrate crypto-location technologies on. Indeed, for some use cases (such as micro-transactions leveraged by IoT devices), end users may wish to use a platform that \textit{does not} charge fees for each transaction. If one is forced to use Proof of Location systems exclusively on the Ethereum platform, they must face the additional overhead of not just paying fees to use the crypto-location network, but also fees to execute the smart contract on the underlying platform.

\section{The XY Oracle Network}

Location data sits quietly at the cornerstone of every part of our daily lives. Its use has increased dramatically over the last decade and now it is so ubiquitously depended on that its disappearance would be catastrophic. The reliance on location data will increase even more in the future. Tomorrow's reliance on location data will, without question, eclipse our current levels of reliance with the emergence of self-driving vehicles, package-delivering drones and smart cities that develop themselves. With the emergence of these location-reliant technologies, our lives will be in the hands of machines, and our safety will sit in direct proportion to the accuracy and validity the location the systems contain. Securing and creating a trustless source of location information will be crucial to transitioning to the world of tomorrow.

Location data has, for the most part, been provided by centralized sources of truth. History shows that such sources are susceptible to interference, vulnerable to attack and, with human-carrying technologies, can soon be fatal. Blockchain technologies, with their decentralized foundations, play a critical role in creating location-secure systems necessary for these technologies. Decentralizing location determination using a network of interconnected devices allows for a significant paradigm change in how we can source location data. Using blockchain technology to record these location transactions makes the system secure, transparent, and reliable.

Within the world of blockchain there exists smart contracts which enables the automated execution of agreements, which removes the reliance on a trusted third party to facilitate every transaction.

The data that smart contracts rely upon (called an \textit{oracle}), must have a high degree of accuracy and be verifiable. The system that records and delivers this data must be protected from the interference, attacks and error. Most importantly, the reported signals must be locked securely and publicly in time for accountability later on, which stands as perhaps blockchain's strongest properties.

%STOPPED HERE
We propose that the need for a full featured, fully decentralized and fully secured crypto-location network will be required to move us from the world of today, to the world of tomorrow. We set about achieving this with a network of technologies called The XY Oracle Network (\textit{XYO Network}). The XYO Network contains four system components, which we are detailed in this paper: Sentinels, Bridges, Archivists and Diviners. These components serve as the underpinning of an ecosystem of connected devices that enable layered location verification, across many devices, as well as many different classes of devices: Bluetooth beacons (including XY's crypto-location enabled Bluetooth device \textit{XY4+}, GPS beacons (including XY's crypto-location enabled GPS device \textit{XYGPS}), cellular, Low-Power Wide-Area Network (including XY's crypto-location enabled LoRa device \textit{XYLoRa}), smartphones, mobile applications, QR-code reading cameras, IoT devices (including doorbells, refrigerators and smart speakers) and even Low Earth Orbit ("LEO") satellites (including XY's LEO satellite, \textit{The SatoshiXY}) and more. This network of devices make it possible to determine if an object is at a specific XY-coordinate at a given time, with the most provable, trustless certainty possible. Underlying these components sits a breakthrough in IoT device security, called Proof of Origin. Tying the XYO Network's economics together are novel cryptoeconomic incentives that ensures each participant acts in accordance with the idealized-state of the XYO Network.

\textbf{We propose that the most important advancement necessary to bridging the present with the future rests on our ability to trust machines. This trust is best achieved through innovations in blockchain technology, and must be made available through the creation of a crypto-location \gls{oracle} network that is resistant to attack and achieves unprecedented \gls{accuracy} and \gls{certainty} with the given restraints of the system.} Once a location oracle network is established, all other real-world \glspl{heuristic} can be accessed as oracle data, creating a full oracle network that provides the highest confidence and accuracy possible for the technologies of tomorrow (self-driving cars, package-carrying drones, as well as others).

\subsection {Meet The Only Cryptographic Location Protocol Built For The World of Tomorrow}
With the advent of blockchain-based, trustless \glspl{smart-contract}, the need for \gls{oracle} services that arbitrate the outcome of a contract grows proportionately. Most current implementations of smart contracts rely on a single or aggregated set of authoritative oracles to settle the outcome of the contract. In cases where both parties can agree on the authority and incorruptibility of the specified oracle, this is sufficient. \textbf{However, in many cases, either an appropriate oracle does not exist or the oracle cannot be considered authoritative because of the possibility of error or corruption.}

Location oracles fall into this category. The divination of the location of a physical world item relies on the reporting, relay, storage, and processing components of the given oracle, all of which introduce error and can be corrupted. Risks include data manipulation, data pollution, data loss, and collusion. Thus the following law exists at the intersection of blockchain and location: \textit{Both \gls{certainty} and \gls{accuracy} of the location are negatively impacted by the lack of a trustless decentralized location oracle.}


\subsection{Relaying Data: Proof of Origin Chains}

\subsection {XYO Network: Core Issues}
The \Gls{xyo-network} provides location data and verification by addressing the following issues: security, accuracy, \gls{certainty}, cost, verifiable trustlesslness and performance.

\subsection {XYO Network: Secondary Issues}
\subsubsection{Privacy}

Similar to Bitcoin, and most blockchain technologies, the most compelling property of blockchain is the built-in accountability that arises from the fully-public ledger. This derives from the fact that each transaction is completely open and viewable. Bitcoin can be understood as a platform that is \textit{anonymous}, but not \textit{private}. The \Gls{xyo-network} shares these traditional blockchain properties; yet, being that location data is sensitive in nature, additional thought for how privacy concerns are handled becomes a necessity. For this reason the XYO Network is built with privacy at the forefront of how its platform runs.

The \Gls{xyo-network} is voluntary. Meaning, if one wishes track an item, or deploy Sentinels, Bridges or Archivists used in verifying the location of items (in exchange for XYO Token), each participant must opt-in to the network. If one does not wish to participate or have any item's location verified, then they can elect not to partipate. It is very straightforward, yet has become a rather surprisingly unique practice.  This gives one more control over their privacy compared to the choices they seem to have today. involving user privacy than what is currently used today (location tracking that is opt-out by default). It is critical that the usage of XYO Network data be voluntary because the \Gls{xyo-network} stores all the Ledger Chains as public data in the \Glspl{archivist}. This creates the possibility of inferred data that is associated with people or things to be used nefariously.

% Zero-knowledge proof explanation
The XYO Network employs a cryptographic tool called zero-knowledge proofs, which are perhaps one of the most powerful tools cryptographers have ever devised. Zero-knowledge proofs provide a method of authentication where no private data is exchanged, which means they cannot be stolen. This is novel because it provides an extra layer of security to not just information being transmitted in real-time, but also data stored on the blockchain ledger for future use.

\begin{displayquote}\textit{``Zero-knowledge proofs may be the future of private trade''}

\vspace{2mm}
---Edward Snowden
\end{displayquote}

It's important to note that location information about you, as well as your devices, is already being compiled in a non-decentralized manner; the key difference is that the data stored is \textit{not anonymous}, but tied to your identity. The XY Oracle Network focuses on making location not just trustless and decentralized, but also on being \textit{identityless}; this is achieved by combining zero-knowledge proof with a cryptographic method we call Proof of Origin, as well as other technologies (which will be covered later on).

Yet in addition to the identityless composition of the XYO Network, there is an additional layer of privacy protection in that the XYO Network is decentralized. A decentralized network removes the profit motive which can motivate malicious actors to build user profiles without permission. Since the data is publicly accessible, there's no incentive to profiteer by accessing and selling information. This is made possible due to the identityless nature of the data comprising the XYO Netowork ledgers.

\subsubsection {Physical Device Security}
Physical device security remains an important, yet tertiary focus of the XYO Network. We believe that securing the integrity of a location device (be it a Sentinel, Bridge or other) is an important component of the XYO Network; however, just as Bitcoin does not involve itself in the detailed technologies of miners, so to sits the nature of XYO Network. Indeed, it falls outside the scope of the XYO Network. Rather, the more important concern of the XYO Network rests in devising the proper cryptoeconomics to encourage valid data, and discourage fraudulent data. Devices to deter malicious actors may involve a combination of hardware innovation and give rise to new industries.

\subsubsection {Decentralized Data Storage}
TODO(Arie): Describe how running a decentralized data storage business is outside the scope of the XYO Network, and how the system is built to support current solutions in the space.

\subsubsection {Replacing Location Protocol Standards}
TODO: We are interested in the most important location problems in our world tomorrow: secure, accurate and certain location signals. We are not interested in replacing protocol standards (such as Latitude, Longitude Coordinates); yet we will be inclusive of the new location protocols that will likely be released in the years to come. We intend to build upon Proof of Location protocols and contribute to them, as they are a useful spoke in the wheel of the XYO Network because they provide additional signals to be analyzed for truthiness.

% TODO(Arie): What does this mean?
\subsection {Assumptions}
Edge devices on the \Gls{xyo-network} that produce and relay the needed location data are not physically secure, can be produced and added to the system at low cost, and are highly resource constrained.

\printglossaries

%*******************************
%**** End Glossary Section *****
%*******************************

\end{document}
