% Preamble
% ---
\documentclass{article}

% Packages
% ---
\usepackage{amsmath} % Advanced math typesetting
\usepackage[utf8]{inputenc} % Unicode support (Umlauts etc.)
\usepackage{hyperref} % Add a link to your document
\usepackage{graphicx} % Add pictures to your document
\usepackage{listings} % Source code formatting and highlighting
\usepackage{framed} % Source code formatting and highlighting
\usepackage{appendix} % Source code formatting and highlighting
\usepackage{csquotes} % Pretty quotes
\usepackage[automake]{glossaries}
\usepackage{xcolor}
\usepackage{pagecolor}
\usepackage[letterpaper, portrait, margin=1.5in]{geometry}

\graphicspath{ {images/} }

\makeglossary

%*******************************
%**** Begin Glossary Section *****
%*******************************

\newglossaryentry{sentinel}
{
    name={Sentinel},
    description={A Sentinel is a heuristic witnesses. It observes heuristics and vouches for the certainty and accuracy of them by producing temporal ledgers. The most important aspect of a Sentinel is that it produces ledgers that Diviners can be certain came from the same source by adding Proof of Origin to them}
}

\newglossaryentry{bridge}
{
    name={Bridge},
    description={A Bridge is a heuristic transcriber. It securely relays heuristic ledgers from Sentinels to Diviners. The most important aspect of a Bridge is that a Diviner can be sure that the heuristic ledgers that are received from a Bridge have not been altered in any way. The second most important aspect of a Bridge is that they add an additional Proof of Origin metadata}
}

\newglossaryentry{archivist}
{
    name={Archivist},
    description={An Archivist stores heuristics as a part of the decentralized data set with the goal of having all historical ledgers stored, but without that requirement. Even if some data is lost or becomes temporarily unavailable, the system continues to function, just with reduced accuracy. Archivists also index ledgers so that they can return a string of ledger data if needed. Archivists store raw data only and get paid solely for retrieval of the data. Storage is always free}
}

\newglossaryentry{diviner}
{
    name={Diviner},
    description={A Diviner answers a given query by analyzing historical data that has been stored by the XYO Network. Heuristics stored in the XYO Network must have a high level of Proof of Origin to determine the validity and accuracy of the heuristic. A Diviner obtains and delivers an answer by judging the witness based on its Proof of Origin. Given that the XYO Network is a trustless system, Diviners must be incentivized to provide honest analyses of heuristics. Unlike Sentinels and Bridges, Diviners use Proof of Work to add answers to the blockchain}
}

\newglossaryentry{webble}
{
    name={webble},
    description={Everything in the world is defined spatially by its \textit{X,Y,Z,T} coordinate and nothing can leave that space. Objects are thus confined to ``webbubbles'', or what are referred to as \textit{webbles}.}
}

\newglossaryentry{gas}
{
    name={gas},
    description={The cost of a transaction (i.e. query) in the form of XYO Tokens}
}

\newglossaryentry{proof-of-origin}
{
    name={Proof of Origin},
    description={Proof of Origin is the key to verifying that ledgers flowing into the XYO Network are valid. A unique ID for source of data is not practical since it can be forged. Private key signing is not practical since most parts of the XYO Network are difficult or impossible to physically secure, thus the potential for a bad actor to steal a private key is too feasible. To solve this, XYO Network uses Transient Key Chaining. The benefit of this is that it is impossible to falsify the chain of origin for data. However, once the chain is broken, it is broken forever and cannot be continued, rendering it an island}
}

\newglossaryentry{proof-of-work}
{
    name={Proof of Work},
    description={Proof of Work is a piece of data that satisfies certain requirements, is difficult to produce (i.e. costly, time-consuming), but easy for others to verify. Producing a Proof of Work can be a random process with a low probability of generation so that rigorous trial and error is required on average before a valid Proof of Work is created}
}

\newglossaryentry{bound-witness}
{
    name={Bound Witness},
    description={Bound Witness is a concept achieved by the existence of a bidirectional heuristic. Given that an untrusted source of data for the use of digital contract resolution (an oracle) is not useful, there is a substantial increase in certainty of the data provided by the establishment of such a heuristic. The primary bidirectional heuristic is proximity since both parties can validate the occurrence and range of an interaction by cosigning the interaction. This allows for a zero-knowledge proof that the two nodes were in proximity of each other.}
}

\newglossaryentry{smart-contract}
{
    name={smart contract},
    description={A protocol coined by Nick Szabo before Bitcoin, purportedly in 1994 (which is why some believe him to be Satoshi Nakamoto, the mystical and unknown inventor of Bitcoin). The idea behind smart contracts is to codify a legal agreement in a program and to have decentralized computers execute its terms, instead of humans having to interpret and act on contracts. Smart contracts collapse money (e.g. Ether) and contracts into the same concept. Being that smart contracts are deterministic (like computer programs) and fully transparent and readable, they serve as a powerful way to replace middle-men and brokers}
}

\newglossaryentry{cryptoeconomics}
{
    name={cryptoeconomics},
    plural={cryptoeconomic},
    description={A formal discipline that studies protocols that govern the production, distribution, and consumption of goods and services in a decentralized digital economy. Cryptoeconomics is a practical science that focuses on the design and characterization of these protocols}
}

\newglossaryentry{xyo-network}
{
    name={XYO Network},
    description={XYO Network stands for ``XY Oracle Network.'' It is comprised of the entire system of XYO enabled components/nodes that include Sentinels, Bridges, Archivists, and Diviners. The primary function of the XYO Network is to act as a portal by which digital smart contracts can be executed through real world geo-location confirmations}
}

\newglossaryentry{certainty}
{
    name={certainty},
    description={A measure of the likelihood that a data point or heuristic is free from corruption or tampering}
}

\newglossaryentry{accuracy}
{
    name={accuracy},
    description={A measure of confidence that a data point or heuristic is within a specific margin of error}
}

\newglossaryentry{oracle}
{
    name={oracle},
    description={A part of a DApp (decentralized application) system that is responsible for resolving a digital contract by providing an answer with accuracy and certainty. The term ``oracle'' originates from cryptography where it signifies a truly random source (e.g. of a random number). This provides the necessary gate from a crypto equation to the world beyond. Oracles feed smart contracts information from beyond the chain (the real world, or off-chain). Oracles are interfaces from the digital world to the real world. As a morbid example, consider a contract for a Last Will \& Testament. A Will's terms are executed upon confirmation that the testator is deceased. An oracle service could be built to trigger a Will by compiling and aggregating relevant data from official sources. The oracle could then be used as a feed or end-point for a smart contract to call out to in order to check whether or not the person is deceased}
}

\newglossaryentry{heuristic}
{
    name={heuristic},
    description={A data point about the real world relative to the position of a Sentinel (proximity, temperature, light, motion, etc...)}
}

\newglossaryentry{transient-key-chain}
{
    name={Transient Key Chain},
    description={A Transient Key Chain links a series of data packets using Transient Key Cryptography}
}

\newglossaryentry{best-answer}
{
    name={Best Answer},
    description={We define the Best Answer as the single answer, amongst a list of Answer Candidates, that returns the highest validity score and has a higher accuracy score than the minimum required accuracy.}
}

\newglossaryentry{best-answer-score}
{
    name={Best Answer Score},
    description={A score generated by a Best Answer Algorithm that ranks the quality of the score.  The higher the score, the better it is, per the algorithm.  This score is used to determine which answer is better given two analyzed answers}
}

\newglossaryentry{best-answer-algorithm}
{
    name={Best Answer Algorithm},
    description={An algorithm used to generate Best Answer Scores when a Diviner chooses an answer.  The XYO Network permits the addition of specialized algorithms and allows the customer to specify which algorithm to use.  It is required that this algorithm will result in the same score when run on any Diviner given the same data set}
}

\newglossaryentry{origin-chain-score}
{
    name={Origin Chain Score},
    description={The score assigned to an Origin Chain to determine its credibility. This assessment takes length, tangle, overlap, and redundancy into consideration}
}

\newglossaryentry{proof-of-origin-chain}
{
    name={Proof of Origin Chain},
    description={A Transient Key Chain that links together a series of Bound Witness heuristic ledger entries}
}

\newglossaryentry{origin-tree}
{
    name={Origin Tree},
    description={A data set of ledger entries taken from various Origin Chains to establish the origin of a heuristic ledger entry with a specified level of certainty}
}

\newglossaryentry{crypto-location}
{
    name={crypto-location},
    description={The realm of cryptographic location technology}
}

\newglossaryentry{trustless}
{
    name={trustless},
    description={A characteristic where all parties in a system can reach a consensus on what the canonical truth is. Power and trust is distributed (or shared) among the network’s stakeholders (e.g. developers, miners, and consumers), rather than concentrated in a single individual or entity (e.g. banks, governments, and financial institutions). This is a common term that can be easily misunderstood. Blockchains don’t actually eliminate trust. What they do is minimize the amount of trust required from any single actor in the system. They do this by distributing trust among different actors in the system via an economic game that incentivizes actors to cooperate with the rules defined by the protocol}
}

\newglossaryentry{xyomainchain}
{
    name={XYOMainChain},
    description={An immutable blockchain in the XYO Network that stores query transactions along with data gathered from Diviners and their associated origin score}
}

\newglossaryentry{xyo-miner}
{
    name={XYO Miner},
    description={Sentinels, Bridges, Archivists, and Ddiviners who take part in answering queries to the XYO Network in an XYO crypto-location mining pool}
}

\newglossaryentry{ideal-state}   
{
    name={ideal-state},
    description={The location-verification standard in an XYO crypto-location mining pool. It can be voted on amongst other XYO miners in the XYO Network system votes to increase or decrease this standard}
}
    
\newacronym{xy-oracle-network}{XY Oracle Network}{XYO Network}


\definecolor{lightgreen}{rgb}{0.95,1,0.95}

\title {XYO Network: Business Primer and Token Economics}

\author{
    Arie Trouw
        \thanks{XYO Network, \texttt{arie.trouw@xyo.network}}
    , Markus Levin
        \thanks{XYO Network, \texttt{markus.levin@xyo.network}}
    , Scott Scheper
        \thanks{XYO Network, \texttt{scott.scheper@xyo.network}}
}

\date{January 2018}

\begin{document}

\pagecolor{lightgreen}

\maketitle

\begin{center}
\line(1,0){50}
\end{center}

\section{Introduction}
In 2013, a groundbreaking cryptographic technology was introduced to the world: a platform called \textit{Ethereum}. A core component of Ethereum is a concept called a \gls{smart-contract}, which reduces a payment and an agreement into lines of code. Imagine if a contract wasn't written on a piece of paper and signed by hand, but instead was written in computer code and only executed when certain conditions were met. Smart contracts empower the world with digital transactions that are deterministically executed by decentralized nodes traveling around the globe.

Let us apply this concept to the world of sports wagering. Take for instance, the following wager between two agents: \textit{Agent A} wishes to wager with \textit{Agent B} that \textit{Team A} will beat \textit{Team B} in a game. Currently, there is no other choice but to employ a trusted, disinterested third party to act as an intermediary for the transaction (in exchange for a fee). This is precisely how the eCommerce world worked before the introduction of Bitcoin. With the innovation of Ethereum, one can now program a smart contract wherein funds from the agent who bet on the losing team, are deposited automatically to the agent who bet on the winning team. This can be done by developing a smart contract to deterministically execute at a specific time in the future (\texttt{block.timestamp}). In order to determine whether \textit{Team A} or \textit{Team B} won, the contract must call out to a data source (such as a website that lists the final scores) after the game concludes. In the world of smart contracts, this external data source is known as an \gls{oracle}. The oracle exists its as the weak point in this system, as external data sources can be hacked (for instance if \textit{Agent A} works for the data source that the smart contract relied upon, he or she could use their privileged access to manipulate or tamper with the data source in order to win the wager, even if the actual results were contradictory).

Data tampering is tempting when a party is financially incentivized to do so, which is why \gls{cryptoeconomics} are typically utilized to make such actions economically unviable. The above example does not rely on cryptoeconomics for \gls{certainty}; rather, to protect against this vulnerability, a concept called \textit{consensus} is deployed for oracles. This enhancement requires the smart contract to not only rely on one data source, but multiple sources of data, all of which must agree and achieve consensus on the winner in order for the contract to execute. Creating such a contract enables two parties to transact with their agreement in a peer-to-peer manner, eliminating need for a trusted third party. The notion is astonishingly simple, yet up until this point in history, revolutionary approach was not possible. Indeed, the implications of this are profound, yet not fully apparent today.

Since the advent of Ethereum, the cryptoassets community has experienced rapid-growth in the form of DApp development and protocol improvements. However, up until this point, every platform (including Bitcoin and Ethereum) has focused almost entirely on digital channels (the online world), instead of real world channels (the offline world).

Progress has begun in the physical realm with the introduction of offline-focused cryptographic platforms that concentrate on specific use cases, such as the intersection of blockchain and the Internet of Things (\textit{IoT})\footnote{Including \textit{IOTA} (www.iota.org) and \textit{Hdac} (www.hdac.io)}. In addition, there are efforts being made to develop protocols that concentrate on the intersection of location and blockchain, which are being labeled \textit{Proof of Location}. These platforms and protocols are interesting and worth supporting; furthermore, they are useful components that serve as a spoke in the wheel of the \Gls{xyo-network}.

However, we still find the majority of blockchain technologies confined primarily to the narrow scope of the Internet. Since its founding in 2012, \textit{XY Findables}, the company behind the XYO Network, has been building a location network in order to make the physical world programmable and accessible to developers. In brief, XY has been working towards the concept of enabling developers (such as those writing Ethereum smart contracts) with the power to interact with the real world as if it were an API. This undertaking is a multi-year project that requires the separation of different components into stages.

The importance of crypto-location technologies making their way to multiple platforms should be highlighted before moving on. Thus far, all crypto-location protocols have focused on the Ethereum platform. Yet, there are other compelling blockchain platforms that make strong arguments for their use, especially in specific applications. For this reason, we have built the XYO Network to be platform-agnostic at its inception. Our open-ended architecture ensures that the XYO Network of today will support the blockchain platforms of tomorrow. The XYO Network supports all blockchain platforms that possess smart contract execution\footnote{This includes Ethereum, Bitcoin + RSK, EOS, IOTA, NEO, Stellar, Counterparty, Monax, Dragonchain, Cardano, RChain, Lisk and others}.

Additionally, the current limitation with Proof of Location protocols (and many other blockchain DApps) centers around their complete and total reliance on Ethereum. While we believe that Ethereum will remain a critical platform in the future of blockchain technology\footnote{The XYO Network is a supporter of Vlad Zamfir's, correct-by-construction Proof of Stake consensus protocol, as well as Ethereum's \textit{sharding} clients.}, it is imperative to the XYO Network that end users be given the choice of what blockchain platform they wish to integrate crypto-location technologies with. Indeed, for some use cases (such as micro-transactions leveraged by IoT devices), end users may wish to use a platform that \textit{does not} charge fees for each transaction. If one is forced to use Proof of Location systems exclusively on the Ethereum platform, they must face the additional overhead of not just paying fees to use the crypto-location network, but also fees to execute the smart contract on the underlying platform.

\clearpage

\begin{center}
\line(1,0){50}
\end{center}

\section{Background \& Previous Attempts}

\subsection{Proof of Location}

The concept of provable location has been around since the 1960s, and can even be dated back to the 1940s with ground-based radio-navigation systems, such as LORAN \cite{blanchard-loran}. Today, there are location services that stack multiple mediums of verification on top of one another to create a Proof of Location through triangularization and GPS services. However, these approaches have yet to address the most critical component we face in location technologies today: designing a system that detects fraudulent signals and disincentivizes the spoofing of location data. For this reason, we propose that the most significant crypto-location platform today will be the one that focuses most on proving the origin of physical location signals.

Surprisingly, the concept of applying location verification to blockchain technologies first appeared in September 2016 at Ethereum's DevCon 2. It was introduced by Lefteris Karapetsas, an Ethereum developer from Berlin. Karapetsas' project, \textit{Sikorka}, enabled \glspl{smart-contract} to be deployed on the spot in the real world, using what he termed, ``\textit{Proof of Presence}.'' His application of bridging location and the world of blockchain focused primarily on augmented reality use cases; and he introduced novel concepts such as challenge questions in proving one's location \cite{karapetsas-sikorka}.

%TODO(Christine - DONE) %https://www.reddit.com/r/ethereum/comments/539o9c/proof_of_location/
On September 17th, 2016, the term, ``\textit{Proof of Location},'' formally surfaced in Ethereum's community \cite{diferrante-proofoflocation}. It was then further expounded upon by Ethereum Foundation developer, Matt Di Ferrante:

\begin{displayquote}\textit{``Proof of Location you can trust is honestly one of the most difficult things to implement. Even if you have many participants that can attest each other's location, there's no guarantee that they wouldn't just go sybil at any point in the future, and since you're always only relying on majority reporting it's a huge weakness.
If you could require some type of specialized hardware device that has anti-tamper tech such that the private key is destroyed when one attempts to open it or change the firmware on it then you could possibly have greater security, but at the same time, it's not like it's impossible to spoof GPS signals either.
A proper implementation of this requires so much fallback and so many different data sources to have any assurance of accuracy, it would have to be very well funded project.''} \cite{diferrante-proofoflocation}

\vspace{2mm}
---Matt Di Ferrante, Developer, Ethereum Foundation
\end{displayquote}


\subsection {Proof of Location: Shortcomings}
% The Importance of Location Being Platform-agnostic:

In summary, Proof of Location can be understood as leveraging the blockchain's powerful properties, such as time-stamping and decentralization, and combining them with devices that are hard to trick. Similar to how the weakness of smart contracts centers around oracles that use a single source of truth (and thus have a single source of failure), crypto-location systems face the same problem. The vulnerability in current crypto-location technologies revolves around the devices that report back an object's location. In smart contracts, this data source is an oracle. The true innovation at the core of the \Gls{xyo-network} centers around a location-based proof underlying the components of our system to create a secure, crypto-location protocol.

\begin{center}
\line(1,0){50}
\end{center}

\section{The XY Oracle Network}

Location data sits quietly at the cornerstone of every part of our daily lives. Its use has increased dramatically over the last decade and is now so ubiquitously relied upon that its disappearance would be catastrophic. The direction of tomorrow's technology is quickly approaching a world with self-driving vehicles, package-delivery drones and smart cities that develop and run themselves. Considering these imminent innovations makes it glaringly evident that our dependence on location data will, without question, eclipse our current usage by an unsurmountable magnitude. With the emergence of these location-reliant technologies, our lives will be in the hands of machines, and our safety will sit in direct proportion to the \gls{accuracy} and validity of the location data used by these new systems. Securing and creating a \gls{trustless} source of location information will be crucial to successfully transitioning to the world of tomorrow.

Location data has predominantly been provided by centralized sources of truth. History has proven that such sources are susceptible to interference, vulnerable to attack and, in the hands of malicious humans, can be fatal. The decentralized infrastructure of Blockchain technology plays a critical role in creating location-secure systems. Decentralizing confirmation of location by using a network of interconnected devices allows for a significant paradigm shift in how the world can source location data. Utilizing blockchain technology to verify and record location data makes location-reliant systems secure, transparent, and reliable.

Blockchain platforms have the ability to facilitate smart contracts which enable the automated execution of agreements. This eliminates the dependence of a trusted third party to facilitate every transaction.

The data that smart contracts rely on (\textit{\glspl{oracle}}), must be verifiable and have a high degree of accuracy. The systems that record and deliver this data must protected from any interference, attacks and/or error. Most importantly, the reported signals sending this data must be locked securely and in time for public accountability later on. These requirements are all fulfilled through the unique and robust properties of blockchain technology.

%STOPPED HERE
We propose that the existence of a full featured, entirely decentralized and highly secured crypto-location network will be absolutely essential in moving the world from the technologies of today, to those of tomorrow. We set about achieving this with a network of technologies called the XY Oracle Network (\textit{\Gls{xyo-network}}). The XYO Network contains four system components, which are detailed in this paper: \Glspl{sentinel}, \Glspl{bridge}, \Glspl{archivist} and \Glspl{diviner}. These components serve as the underpinnings of an ecosystem of connected devices that enable layered location verification across a large volume of various classes of devices: Bluetooth beacons (including XY's crypto-location enabled Bluetooth device \textit{XY4+}, GPS beacons (including XY's crypto-location enabled GPS device \textit{XYGPS}), Low-Power Wide-Area Network devices (including XY's crypto-location enabled LoRa device \textit{XYLoRa}), mobile devices, mobile applications, QR-code reading cameras, IoT devices (including smart doorbells, appliances and speakers), Low Earth Orbit (``LEO'') satellites (including XY's LEO satellite, \textit{The SatoshiXY}) and more. This network of devices makes it possible to determine if an object is at a specific XY-coordinate at a given time, with the most provable, trustless \gls{certainty} possible. At the core of the XYO Network's four components is a true breakthrough in IoT device security, called \Gls{proof-of-origin}. The economic framework of the XYO Network is held together by novel \glspl{cryptoeconomics} incentives that ensure each participant acts in accordance with the ideal-state of the XYO Network.

\textbf{We propose that the most important advancement necessary for bridging the present to the future rests on the world's ability to trust machines. This trust is best achieved through innovations in blockchain technology, and must be made available through the creation of a crypto-location oracle network that is resistant to attack and achieves unprecedented accuracy and certainty within the given restraints of the system.} Once a location oracle network is established, all other real-world \glspl{heuristic} can be accessed as oracle data, creating a full oracle network that provides the highest confidence and accuracy necessary for the proliferation of the technologies of tomorrow (self-driving cars, package-carrying drones, as well as others).

\subsection {Meet The Only Cryptographic Location Protocol Built For The World of Tomorrow}
With the advent of blockchain-based, \gls{trustless} \glspl{smart-contract}, the need for \gls{oracle} services that arbitrate the outcome of a contract grows proportionately. Most current implementations of smart contracts rely on a single or aggregated set of authoritative oracles to settle the outcome of a contract. In cases where both parties can agree on the authority and incorruptibility of the specified oracle, this is sufficient. \textbf{However, in many cases, either a sufficient oracle does not exist or the oracle cannot be considered authoritative due to the possibility of error or corruption.}

Location oracles fall into this category. The divination of the location of a physical world item relies on the reporting, relay, storage, and processing components of the given oracle, all of which introduce error and can be corrupted. Risks include data manipulation, data pollution, data loss, and collusion. Thus the following law exists at the intersection of blockchain technology and location data: \textbf{Both the \gls{certainty} and \gls{accuracy} of location are negatively impacted by the lack of a \gls{trustless}, decentralized location oracle.}

\subsection{Privacy: Applying Zero-knowledge Proof to Location Data}

Similar to Bitcoin, and most blockchain technologies, the most compelling property of blockchain is the built-in accountability that is inherent of a fully-public ledger. This derives from the fact that each transaction is completely open and viewable. Bitcoin can be interpreted as a platform that is \textit{anonymous}, but not \textit{private}. The \Gls{xyo-network} shares these traditional blockchain properties; yet, since location data is sensitive in nature, additional thought in how privacy concerns are handled becomes a necessity. For this reason, the XYO Network is built with privacy at the forefront of how its platform runs.

The XYO Network is voluntary. Meaning, if one wishes track an item, or deploy \Glspl{sentinel}, \Glspl{bridge}, or \Glspl{archivist} to aid in verifying the location of items (in exchange for XYO Tokens), one must opt-in to the network. If one does not wish to participate or have any item's location verified, then they can elect not to participate. Thus, the XYO Network gives one more control over their privacy compared to platforms that have mandatory opt-in terms and conditions. It is critical that the participation in and usage of the XYO Network be voluntary since the \Gls{xyo-network} stores all Ledger Chains in the Archivists as public data. This creates the possibility of inferred data that can be associated with people or things to be used nefariously.

% Zero-knowledge proof explanation
The XYO Network utilizes a cryptographic method called a zero-knowledge proof, which is perhaps one of the most powerful tools cryptographers have ever devised. Zero-knowledge proofs provide authentication without exchanging private data, which means private data cannot be exposed or stolen. This is a novel advancement because it provides an extra layer of security not only to information transmitted in real-time, but also data stored on the blockchain ledger for future use.

\begin{displayquote}\textit{``Zero-knowledge proofs may be the future of private trade''} \cite{snowden-privatetrade}

\vspace{2mm}
---Edward Snowden
\end{displayquote}

It's important to note that location information about everyone and their devices, is already being compiled in a centralized manner; the key difference is that the data stored is not \textit{anonymous}, but tied to their identity. The XYO Network focuses on making location not only \gls{trustless} and decentralized, but also \textit{identityless}. This is achieved by combining a zero-knowledge proof with a cryptographic method we call \Gls{proof-of-origin}, as well as other technologies which will be covered later on.

In addition to the identityless composition of the XYO Network, there is an additional layer of privacy protection implied by the decentralized architecture of the XYO Network. A decentralized network eliminates the motive to profit from transactions, which could otherwise encourage malicious actors to build fake user profiles without permission. Since the data is publicly accessible, there is no incentive to profit by accessing and selling information. This is made possible by virtue of the identityless nature of the data comprising the XYO Network.

\begin{center}
\line(1,0){50}
\end{center}

\section{Applications}

From straightforward to complex, the \Gls{xyo-network}'s usage  has vast applications that span a multitude of industries. For example, take an eCommerce company that could offer its premium customers payment-upon-delivery services. To be able to offer this service, the eCommerce company would leverage the \Gls{xyo-network} and XY Platform (which uses XYO Tokens) to write a \gls{smart-contract} (i.e. on Ethereum's platform). The XYO Network could then track the location of the package being sent to the consumer along every single step of fulfillment; from the warehouse shelf to the the shipping courier, all the way into the consumer's house and every location in between. This could enable eCommerce retailers and websites to verify, in a \gls{trustless} way, that the package not only appeared on the customer's doorstep, but also safely inside their home. Once the package is confirmed to be in the customer's home (defined and verified by a specific XY-Coordinate), the shipment is considered complete and the payment to the vendor gets released. The eCommerce integration of the XYO Network thusly enables the ability to protect the merchant from fraud as well as ensure consumers only pay for goods that arrive in their home.

Consider an entirely different integration of the XYO Network with a hotel review site, whose current problem is that their reviews are often not trusted. Hotel owners are intrinsically incentivized to improve their reviews at any cost. What if one could say with extremely high \gls{certainty} that someone was in San Diego, flew to a hotel in Bali and stayed there for two weeks, returned to San Diego, and then wrote a review about their hotel stay in Bali? The review would have a very high reputation, especially if it was written by a serial reviewer who has written many reviews with verified location data.

The growing expansion of platforms and services that tie the online world to the physical world require equally expansive solutions to their inevitable complications. The solutions the XYO Network can provide are infinite and its potential impact in the world is unlimited.

\subsection{eCommerce}

According to a recent study released by Comcast, more than 30\% of Americans have had a package stolen from their porch or doorstep \cite{comcast-packagesurvey}. As the market share of eCommerce continues to grow, this problem will only become more prevalent. Megasites like Amazon are experimenting with different solutions to offer confirmed secure delivery as a premium service to their customers.

By utilizing the \Gls{xyo-network} and XYO Tokens, companies like Amazon and UPS can offer, as a premium service, an independently confirmed ledger to track every step of a shipment's progress, starting at the fulfillment center and ending with the package's secure delivery within the customer's home. As a \gls{trustless} and decentralized system, the XYO Network provides independent confirmation not only of a package's delivery, but of its entire shipping history. This also allows a retailer or eCommerce site to offer payment-upon-delivery, utilizing a \gls{smart-contract} to protect the merchant from fraud or loss.

When a customer finalizes an order, a smart contract is created which will release payment to the merchant upon successful delivery of the purchased product. The shipment will include an XYO Network \Gls{sentinel}, a low-cost electronic device that records its interactions with other XYO Network devices on its blockchain ledger. Other devices on the XYO Network will likewise record their interactions with other packages being shipped. Every one of these interactions will be independently verifiable, asserting a web of locational \gls{certainty} that stratches all the way back to the shipment's point of origin. When the shipment reaches its destination (as confirmed by its interaction with XYO Network devices within the buyer's home), the smart contract will be executed and the payment will be released. Should there be a dispute, the ledger will provide a history that can confirm the delivery of the shipment or show where it went off track.

The terminal point of the transaction - the point where the package is delivered and payment is released - will be determined at the time the order is placed. Amazon has experimented with multiple secure delivery systems, including lockers in public places like convenience stores and even electronic locks that give their delivery team access to customers' homes. XYO Network devices within these secure locations will confirm delivery. In an Amazon locker, the shipped package will interact not just with its locker, but with XYO Network devices in other lockers and the customers that use them. In the customer's home, the XYO Network nodes could include the customer's phone, IoT devices, and even the Amazon Echo that was used to place the order.

\subsection{Hospitals and Medical Errors}
Medical errors are the third leading cause of death in the United States, according to a study released by the Johns Hopkins School of Medicine \cite{makary-medicalerrors}. Many of these preventable deaths are a result of operational or record-keeping errors, including adverse drug interactions, improper medical records, and even unnecessary surgeries. In a letter to the Centers for Disease Control and Prevention, the study's author, Dr. Martin Makary, stated:

\begin{displayquote}\textit{``It is time for the country to invest in medical quality and patient safety proportional to the mortality burden it bears. This would [include] research in technology that reduces harmful and unwarranted variation in medical care.''} \cite{makary-johnshopkins}

\vspace{2mm}
---Dr. Martin Makary
\end{displayquote}

By tying the \Gls{xyo-network} into the operational frameworks that are already in place in Hospitals, care providers can significantly reduce failures in communication and record keeping that result in patient injury and death. Utilizing the XYO Network and XYO Tokens can provide a \gls{trustless}, decentralized, and independently verifiable record of all patient interactions with any staff as well as a log of relevant patient data such as the patient's vitals, treatment details, and test results for the duration of their stay.

The XYO Network is a web of devices that record and archive \gls{heuristic} data using a blockchain ledger. Whenever a device on the XYO Network interacts with another XYO Network device, it logs this interaction. By reviewing this ledger of interactions and the additional data it provides, it is possible to verify with a high degree of \gls{certainty} that a specific interaction happened at a specific time in a specific location.

For example, imagine a patient, John Doe, who is admitted to the E.R. John is given an identification bracelet that is also an XYO Network \Gls{sentinel}, which keeps a record of any XYO Network devices John interacts with. The monitor that reads John's vital signs is also a Sentinel. It logs John's vitals as heuristic data, and the communication between the two devices eliminates the potential for human error in record keeping. The monitor also serves as an XYO Network \Gls{bridge}, reporting and archiving the blockchain ledgers of any Sentinels it interacts with.

When John is treated by a doctor or nurse, these interactions are recorded on John's ledger, the monitor's ledger, and the ledger of a \Gls{sentinel} embedded in the staff member's hospital ID. The XYO Network could even keep a log of medications John receives, and because a Sentinel could be linked to the medication itself, it could provide confirmation that the correct dosage of the correct medication was administered, confirming the \gls{accuracy} of John's medical record.

\begin{center}
\line(1,0){50}
\end{center}

\section{XY Findables}

The \Gls{xyo-network} will be built upon an existing infrastructure of 1,000,000 devices that have been distributed throughout the world via our consumer-facing business, XY Findables. XY's Bluetooth and GPS devices allow everyday consumers to place physical tracking beacons on the things they want to keep track of (such as keys, luggage, bikes and even pets). If they misplace or lose such items, they can see exactly where they are by viewing their location on a smartphone application. In just six years, XY has created one of the largest consumer Bluetooth and GPS networks in the world.

We are fortunate to have a consumer business that has successfully built this real-world network. Most location networks fail to reach this phase and attain the critical mass necessary to build out an extensive network. However, the \Gls{sentinel} network we have established is only the starting point. The XYO Network is an open system that any operator of location devices can plug into and begin earning XYO Tokens.

Generally, the greater the Sentinel cardinality in the XYO Network, the more reliable the network. To further grow its network, the \Gls{xyo-network} is engaging with other businesses to expand its network of Sentinels beyond its own network of XY beacons.

\begin{center}
\line(1,0){50}
\end{center}

\section{Our Team}
XY's team is comprised of seasoned engineers, business development professionals and marketing experts. Arie Trouw solely founded XY Findables in 2012. Scott Scheper and Markus Levin joined as co-founders of the blockchain initiative in 2017 to assist in building the XY Oracle Network.

\subsection{Founders}

\begin {framed}
\begin {center}
\textbf{Arie Trouw - Founder - Architect}\par
\end {center}
Ten years before Elon Musk wrote his first line of computer code, another young prodigy from South Africa was busy writing software on his TRS-80 Model I. In 1978, at the age of 10, Arie Trouw started developing software on the TRS-80 Model I, moving on to Atari, Apple, and PC. He then ran a series of bulletin boards centering on game-theory modification.

Arie is an accomplished serial entrepreneur with a rich history of technological breakthroughs and business successes involving multiple 8-figure exit events. He is a strong believer in decentralization and the creation of the integrated owner/user model. Arie founded XY in 2012 (incorporated as Ength Degree, LLC before it was converted to a C Corporation in 2016).

He currently serves as Chief Executive Officer, Chief Financial Officer, Chief Operating Officer and Chairman of the Board of Directors. Prior to starting XY-The Findables Company, Arie was CEO and Chairman of Pike Holdings Inc and Chief Technology Officer of Tight Line Technologies LLC. He received his Bachelor of Science in Computer Science from the New York Institute of Technology. Fun Fact: He is a member of one of the first Afrikaans speaking families to emigrate to the US from South Africa in 1976.

\end {framed}

\begin {framed}
\begin {center}
\textbf{Markus Levin - Co-Founder - Head of Operations}\par
\end {center}
Markus mined his first Bitcoin in 2013 and has been captivated by blockchain technologies ever since. Markus has over 15 years experience in building, managing and growing companies around the globe. Markus is originally from Germany (with English as his second language), and specializes in getting the most out of companies by implementing data-driven systems and utilizing the key talents of each employee to get the best out of his team.

After dropping out of his Ph.D. studies at Bocconi University, Markus began working with companies in hyper-growth industries around the globe. Markus has led cutting-edge technology ventures such as Novacore, ``sterkly'' (yes, with a lowercase ``s''), Hive Media and Koiyo.

\end {framed}

\begin {framed}
\begin {center}
\textbf{Scott Scheper - Co-Founder - Head of Marketing}\par
\end {center}
Scott has worked on many exciting ventures with exceptionally talented people, including the Co-founder of Uber. Scott's first ``real boss'' was Arie Trouw, who hired Scott in 2009 during an economic recession, when very few companies were hiring, and even fewer were starting companies. What began as a Facebook app startup with four guys and a ping pong table, grew to over 200 employees and 9-figures in revenue in less than two years.

In 2013 Scott took a break from corporate life to pursue the dream of working remotely on a laptop while sipping tropical drinks on the beaches of St. Thomas, Virgin Islands (U.S.). During this period, Scott launched Greenlamp, a programmatic advertising agency specializing in direct-response media buying. The agency was fully automated; built entirely using algorithms to manage the campaigns. The team was built with project-basis software engineers, and had only one full-time employee: that being Scott. The advertising campaigns were managed by an automated system, nicknamed ``Stewie'' (Family Guy). 24-hours a day, Stewie managed everything, making automated tweaks to the advertising campaigns. He even emailed Scott to chat about the changes made (Stewie's emails included signature Stewie lines). In its first year of operation, Greenlamp generated over \$12M in revenue.

When not working, Scott can be found reading books by his idols, Gary C. Halbert and Charlie Munger, or sometimes even outside with friends and family in San Diego, California.

\end {framed}

\subsection{Directors, Managers, and Supervisors}
\begin {framed}
\begin {center}
\textbf{Christine Sako - Head of Analytics}
\end {center}
\end {framed}

\begin {framed}
\begin {center}
\textbf{Johnny Kolasinski - Head of Media}
\end {center}
\end {framed}

\begin {framed}
\begin {center}
\textbf{Jordan Trouw - Customer Experience Manager}
\end {center}
\end {framed}

\begin {framed}
\begin {center}
\textbf{Lee Kohse - Sr. Design Engineer}
\end {center}
\end {framed}

\begin {framed}
\begin {center}
\textbf{Louie Tajeda - Warehouse Logistics Supervisor}
\end {center}
\end {framed}


\begin {framed}
\begin {center}
\textbf{Maria Cornejo - Retail Management Supervisor}
\end {center}
\end {framed}

\begin {framed}
\begin {center}
\textbf{Maryann Cummings - Client Support Manager}
\end {center}
\end {framed}

\begin {framed}
\begin {center}
\textbf{Patrick Turpin - Hardware QA Supervisor}
\end {center}
\end {framed}

\begin {framed}
\begin {center}
\textbf{Vicky Knapp - Sr. Accounting Manager}
\end {center}
\end {framed}

\begin {framed}
\begin {center}
\textbf{William Long - Head of Hardware}
\end {center}
\end {framed}

\begin{center}
\line(1,0){50}
\end{center}

\section{Token Economy}

The \Gls{xyo-network} will rely on an ERC20 token called the XYO Token used to incentivize the desired behavior of providing accurate, reliable location. XYO Tokens can be thought of as``gas'' needed to interface with the real world in order to verify the XY-coordinate of a specified object.

The process works like this: A token holder first queries the \Gls{xyo-network} with a query (e.g. \textit{``Where is my eCommerce order package with XYO Address} \texttt{0x123456789...}''). The query then gets sent into a queue, where it waits to be processed and answered. A user can set their desired confidence level and XYO gas price at query creation. The cost of a query (in XYO Tokens) is determined by the amount of data required to provide an answer to the query as well as market dynamics.  The more data needed, the more expensive the query and higher the XYO gas price. Queries to the XYO Network have the potential to be very large and expensive. For instance, a trucking and logistics company could query the XYO Network to ask, ``\textit{What is the location of every single car in our fleet?}''

Once the XYO Token holder queries the XYO Network and pays the requested gas, all \Glspl{diviner} working on the task call out to the relevant \Glspl{archivist} to retrieve the pertinent data needed to answer the query. The data returned is derived from the \Glspl{bridge}, who originally gathered the data from the \Glspl{sentinel}. Sentinels are essentially the devices or signals that verify the location of objects. These include entities such as Bluetooth trackers, GPS trackers, geo-location tracking built into IoT devices, satellite tracking technology, QR-code scanners, RFID scanning and many others. XY Findables has pioneered and launched its consumer Bluetooth and GPS business, which has allowed it to test and process real-world location \glspl{heuristic}. All efforts in developing the XY Findables consumer business have served to help significantly in designing the XYO Network Blockchain Protocol.

\begin{center}
\line(1,0){50}
\end{center}

\section{Token Generation Event}

As part of our launch, the \Gls{xyo-network} will hold a Token Generation Event, where we will distribute the first instances of XYO Tokens that can be used to power queries in our platform.

\subsection {XYO Token Specifications}
\begin{itemize}
\item Smart contract platform: Ethereum
\item Contract Type: ERC20
\item Token: XYO
\item Token Name: \Gls{xyo-network} Utility Token
\item Token Address: 0x55296f69f40ea6d20e478533c15a6b08b654e758
\item Total issuance: Finite and capped at the amount reached after the Token Main Sale
\item Amount issued during the main sale: Unlimited
\item Unsold and Unallocated tokens: Burned after the token sale event. No further XYO tokens will be generated after the Main Sale ends.
\end{itemize}

\begin{center}
\line(1,0){50}
\end{center}

\section{Roadmap}
%TODO(Christine): Add the RoadMap section from here 2017 on down
XY has been working towards building an open world of location-verification since 2012 by launching a successful Bluetooth/GPS consumer business critical to understanding and building a real-world location network. Today, XY has over 1,000,000 beacons across the world.

\subsection{2012}
\begin{itemize}
\item \textbf{XY Is Founded}

Arie Trouw develops the idea for XY, a company that focuses on the Internet of Things (IoT) space by concentrating specifically on XY-coordinate data.
\end{itemize}

\subsection{2013}
\begin{itemize}
\item\textbf{XY Launches Consumer-Facing B2B Location Brand for Retail Called ``Webble''}

XY launches ``Webble,'' which soon becomes the largest horizontally integrated hyper-location network. Webble aims to compete with Yelp in providing merchants better tools to interact one-on-one with their customers (eliminating the need for Yelp as a middle-man).

\item\textbf{Webble Network Rolls Out in 9,000 Retail Stores in Southern California}

Webble successfully launches and executes a direct-to-retail location business by distributing Webble SmartSpot stickers on the doors of over 9,000 restaurants and shops throughout San Deigo, CA. This sticker represents the integration of an XY Webble Bluetooth beacon with the business and rewards customers for their loyalty who choose to opt-in to the service. 
\end{itemize}

\subsection{2014}
\begin{itemize}
\item \textbf{XY Establishes Bluetooth Tracker Brand ``XY Find It'' To Build a Larger XY-Network}

XY shifts its focus to direct-to-consumer location technology by releasing the XY Find It brand; taking on the consumer Bluetooth tracking market.

\item \textbf{1st XY Find It Device Developed \& Shipped To The World}

XY successfully launches and releases its very first consumer product: the XY Find It.
\end{itemize}

\subsection{2015}
\begin{itemize}
\item \textbf{XY Launches Its Second Generation Product: The XY2}

XY releases the XY2, the first-ever Bluetooth location device that focuses specifically on range and battery life. By utilizing a replaceable battery, XY sets industry standards and establishes in-device concentric entanglement technology.

\item \textbf{XY Passes 300,000 Devices Sold}

XY successfully scales and rapidly sells the XY2, making it a leading device in its category and generating over \$1.3 million in revenue.
\end{itemize}

\subsection{2016}
\begin{itemize}
\item \textbf{XY Releases Its Third Generation Product: The XY3}

XY launches the XY3, its Bluetooth tracker that introduces feedback enabled two-way Bluetooth location tracking.

\item \textbf{XY Becomes SEC Qualified and Issues Reg A+ Securities}

XY successfully completes SEC qualifications and reporting standards required to offer the sale of its securities and begins accepting investments through the the United States Security \& Exchange Commission's Regulation A+ qualification. To purchase securities in XY's Reg A+ Offering, visit the XY Findables Reg A+ offering website.

\item \textbf{XY Triples Sales Year-Over-Year}

XY's sales continue to rise; the company generates over 3 times its previous year's sales metric performance goals.
\end{itemize}

\subsection{2017}
\begin{itemize}
\item \textbf{XY Releases Groundbreaking GPS Tracking Device: The ``XYGPS''}

XY launches the world's first hybrid GPS and Bluetooth technology enabled device. The XYGPS is able to report its location anywhere in the world where Cellular and GPS data is available.

\item \textbf{XY Releases The XY4+ Device}

XY releases the XY4+ device which is capable of operating as an \Gls{xyo-network} node via firmware update.

\item \textbf{XY Crosses The 1,000,000 Beacon Mark}

The one-millionth XY device is born.

\item \textbf{XY's Blockchain-based Oracle Network Is Born}

Development of moving the internal XY location network platform to an open blockchain implementation begins; the XY Oracle Network is born.
\end{itemize}

\subsection{2018 - Q1 \& Q2 }
\begin{itemize}
\item \textbf{XY Mints The First ``XYO Token'' to Be Used For \Gls{smart-contract} to Access the XY Oracle Network}

The first XYO Token is created and represents the official currency to be used throughout the entire XYO Network.

\item \textbf{XY to Complete XYO on Test Network (``XY TestNet'')}

XY will complete the development of the XYO Testnet and begin rolling out its location-focused blockchain protocol to its \Gls{sentinel} devices.

\end{itemize}

\subsection{2018 - Q3 \& Q4}
\begin{itemize}
\item \textbf{XY to Launch XY Oracle Main Network ("XY MainNet")}

XY will issue a complete roll out of the XYO Network to its XY Sentinel beacons and start tests with new Sentinel partners (specifically IoT companies and mobile app developers).

\item \textbf{XY to Complete API for Smart Contract Developers to Interact With the XYO Network}

Release of the XYO Network API that enables smart contract developers to write contracts to interact with the XYO Network. Libraries to be developed: Ethereum Solidity Library, Ethereum Viper Library and JavaScript library for websites to interact with XY's Oracle Network (similar to the Web3.js integration with MetaMask).

\item \textbf{XY to Release XY Sticker-Based Trackers, Which Can be Added to eCommerce Packages}

Launch the sticker-based tracking product, the ``XY-Stick'' which enables eCommerce retailers to track every single one of their products in real-time.

\end{itemize}

\subsection{2019}
\begin{itemize}
\item \textbf{XY to Grow Global Network of Diversified Location Sentinel Devices}

Grow coverage of XY Sentinels as well as other components of the XYO Network (\Glspl{bridge}, \Glspl{archivist}, and \Glspl{diviner}).

\item \textbf{XY to Onboard Larger Businesses, Organizations and Retail Companies That Have Use-Cases for Location Verification}

Formalize business partnerships with enterprises and larger entities who can benefit from a decentralized, \gls{trustless} location \glspl{oracle} (e.g. logistics systems, supply chain companies, work hour auditors, eCommerce businesses and countless other niches).

\end{itemize}

\subsection{2020+}
\begin{itemize}
\item \textbf{XY to Expand Global Reach of entire XYO Network}
\end{itemize}

\begin{center}
\line(1,0){50}
\end{center}

\section{Cryptoeconomics}

There is an elephant in the room when it comes to modern \gls{cryptoeconomics}: many coins have become more useless than the assets they were seeking to displace (fiat currencies).

The \Gls{xyo-network} believes that the value of a token should remain in direct proportion to its utility, which to some degree relies on the number of transactions it participates in. Many cryptocurrencies today focus almost exclusively on incentived systems that reward miners; they do not focus on building incentives for token users. Over time, this imbalance creates an undesirable ecosystem for every participant involved (miners, token holders and tertiary entities who build upon its platform).

In an XYO crypto-location mining pool there are \textbf{\Glspl{xyo-miner}} (e.g \Glspl{sentinel}, \Glspl{bridge}, \Glspl{archivist}, \Glspl{diviner}) who take part in answering queries to the XYO Network. In this pool, if a majority of XYO Miners are of low quality, the entire pool of XYO Miners can vote to set the location-verification bar low. However, as soon as more competitive machines are introduced to the pool, the system votes to increase its \textbf{\gls{ideal-state}} for the system. Thus, instead of relying on the computing technology of a few centralized mining pools with access to the most powerful resources, the progression of the XYO mining system stays in direct proportion to the advancements in computing technology of the world.

In any healthy economic token system, there is a balanced ratio of liquidity. However, a vast majority of today's token systems have their pendulums frozen in time at the low end of this metric. In the case of Bitcoin and even Ethereum, a very small minority of mining pools controls the majority of the ecosystem. This creates a problem each token system aims to solve: centralization.

\subsection{Incentivizing Token Usage}

A system in which token holders are encouraged \textit{not} to use their tokens creates a long-term problem for the underlying economy. It creates an ecosystem with very scarce stores of value and triggers a natural impulse to invent reasons for \textit{not} using the token, instead of boosting utility and liquidity. Lack of token liquidity is often ignored by token holders because the artificial scarcity created by reticent token-spending creates short-term spikes, but the question is: \textit{At what cost?}

\textbf{The problem most \glspl{cryptoeconomics} incentives have is that the focus is placed too strongly on the token miners, and not at all on the token users. The XYO Token takes both into account by defining the \gls{ideal-state} and rewarding market participants who hold in-memory accounts of the ideal-state and act upon it being met.}

%Token incentive mechanisms: Token-back rewards, Lottery pools per transaction and more
Depending on the natural flow of the XYO Token economy, a token holder will be rewarded at different points in time with varying token usage incentives: mechanisms such as token rewards for transacting and even leveraging lottery mechanisms.\footnote{The specific token liquidity mechanisms and \% yields for token holders will be outlined in a future paper} In a system where transaction volume is high, a user who preserves the token won't be missing out on transacting. However, similar to how safety measures are in place to prevent fraud amongst miners coming up with wrong answers (which results in XYO Token loss), so too will users be penalized who transact with other parties in a circular manner in order to game the system into receiving liquidity incentives.

The \Gls{xyo-network} provides mechanisms to maintain a healthy economic token system and a balanced liquidity ratio. \Glspl{xyo-miner} are incentivized to not only provide accurate data, but to also know when to provide no data at all. In order to not pollute the ecosystem with inaccurate data, an XYO Miner can pass off the opportunity to a competing XYO Miner (i.e. \Gls{sentinel}, \Gls{archivist}, etc.). The end-user holding XYO Tokens is encouraged to transact more when network liquidity is low, compared to when network liquidity is high. The token user receives economy-based rewards which are given up by XYO Miners who could have computed or verified the data, but elected not to in order to maintain the health of the ecosystem. Essentially, rich machines forfeit the reward they would have received and pass it on to the transacting end-user as well as the second-best machine that took over the task, in order to create a higher quality token system. 

The Bitcoin mining market presents a situation similar to the \textit{prisoner's dilemma} \cite{lave-prisonersdilemma}. As a whole, Bitcoin would benefit more if market participants collaborated to some degree. However, by design of the system, self-interest typically prevails due to simplicity. Adam Smith calls this phenomenon, ``\textit{greatest exactness},'' declaring ``accurate in the highest degree, and admit of no exceptions or modifications, but such as may be ascertained as accurately as the rules themselves, and which generally, indeed, flow from the very same principles with them.''\cite{smith-sentiments} For economies that rely on cognitive beings who are subject to human nature, simplistic, hard rules tend to prevail. Smith understood the natural instinct of humans to operate with \textit{absolute} rules, rather than rules of \textit{negotiation}. He believes this is because holding the ideal-state of a system concurrently, in-memory, is too taxing to the brain. In other words, ``hard-and-fast rules are easier to keep than rules that are slightly relaxed. The opposite should be true.''\cite{roberts-howadamsmith} As a result, current cryptocurrency token economies are inefficient as their tokens do not incentivize participants properly, partly because they are based on economic theory that pre-date blockchain technologies.

The XYO Network addresses these shortcomings and proposes solutions that re-calibrates \glspl{cryptoeconomics} dynamics and revolutionizes blockchain cryptocurrency technology forever.

\begin{center}
\line(1,0){50}
\end{center}

\section {Acknowledgements}
This green paper is the result of our decision to make our white paper more concise. We did this by refining the white paper to contain only the technical details of the \Gls{xyo-network}. We created this green paper to outline the business details, our strategy and the background of blockchain \& location protocols. We thank Raul Jordan (Harvard College, Thiel Fellow and XYO Network Advisor) for his suggestion to compose a separate green paper in the first place. We thank Christine Sako for her exceptional work ethic and attention to detail in her review. After spending a great deal of time and effort structuring our white paper, Christine carried her work even further by applying the same best-practices to our green paper. We thank Johnny Kolasinski for the compilation of use case applications. Last, we thank John Arana for his careful review and creative input for our efforts.


\begin{center}
\line(1,0){50}
\end{center}

\begin{thebibliography}{9}

\bibitem{blanchard-loran}
Blanchard, Walter.
\textit{Hyperbolic Airborne Radio Navigation Aids}.
Journal of Navigation, 44(3), September 1991.

\bibitem{karapetsas-sikorka}
Karapetsas, Lefteris.
\textit{Sikorka.io}.
\\\texttt{http://sikorka.io/files/devcon2.pdf}.
Shanghai, September 29, 2016.

\bibitem{diferrante-proofoflocation}
Di Ferrante, Matt.
\textit{Proof of Location}.
\\\texttt{https://www.reddit.com/r/ethereum/comments/539o9c/proof\_of\_location/}.
September 17, 2016.

\bibitem{snowden-privatetrade}
Snowden, Edward.
\textit{I'm with Vitalik}.
\\\texttt{https://twitter.com/Snowden/status/943164990533578752}
Twitter, December 19, 2017.

\bibitem{comcast-packagesurvey}
Comcast.
\textit{Survey: Nearly One-Third of Americans Have Had Packages Stolen from Their Doorsteps}.
Business Wire, Philadelphia, PA, December 14, 2017.

\bibitem{makary-medicalerrors}
Makary, Martin and Michael Daniel.
\textit{Study Suggests Medical Errors Now Third Leading Cause of Death in the U.S.}
John Hopkins Medicine, May 3, 2016.

\bibitem{makary-johnshopkins}
Makary, Martin.
\textit{Johns Hopkins professor: CDC should list medical errors as 3rd leading cause of death}.
Washington Report, Baltimore, MD, May 4, 2016.

\bibitem{lave-prisonersdilemma}
Lave, Lester B.
\textit{An Empiracle Description of the Prisoner's Dilemma Game}.
\\\texttt{https://www.rand.org/content/dam/rand/pubs/papers/2009/P2091.pdf}.
The RAND Corporation, P-2091, September 14, 1960.

\bibitem{roberts-howadamsmith} 
Russ Roberts. 
Roberts, Russ. 
\textit{How Adam Smith Can Change Your Life}. 
Portfolio / Penguin, New York, NY, October 9, 2014.

\bibitem{bradway-numeraire} 
Bradway, Geoffrey, Richard Craib, Xander Dunn, and Joey Krug.
\textit{Numeraire: A Cryptographic Token for Coordinating Machine Intelligence and Preventing Overfitting}.
\\\texttt{https://numer.ai/whitepaper.pdf}.
February 20, 2017.
 
\bibitem{smith-sentiments} 
Adam Smith
\textit{The Theory of Moral Sentiments}. 
A. Millar, London, 1759.

\end{thebibliography}

\printglossaries

%*******************************
%**** End Glossary Section *****
%*******************************

\end{document}
